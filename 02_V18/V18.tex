\input{../header.tex}

\title{V18:\\ Germanium Detektor}
\author{Benedikt Lütke Lanfer \and Enno Wellmann}
\date{06. Mai 2024}
\publishers{TU Dortmund – Fakultät Physik}

\begin{document}
\begin{titlingpage}
    \begin{center}
        \begin{Huge}
            \textbf{\thetitle\\}
        \end{Huge}
    \end{center}
    \vspace{4cm}
    \includegraphics[width=\textwidth]{Bilder/Logo_TU.png} \\
    \vspace{4cm}
    \begin{center}
        \begin{huge}
            \theauthor\\
        \end{huge}
        \vspace{0.5cm}
        \begin{Large}
            benedikt.luetkelanfer@tu-dortmund.de\\
            enno.wellmann@tu-dortmund.de\\
            \vspace{1.4cm}
            Bearbeitet: \today\\
            Abgabe: \thedate\\
            TU Dortmund – Fakultät Physik\\
        \end{Large}
    \end{center}
\end{titlingpage}
\tableofcontents
\newpage

\section{Zielsetzung}

%---------------------------------------------------------------------------------------------------------------------------------------------------------------%

\section{Theorie}
\subsection{Myonen}
Myonen sind Teilchen des Standardmodells.
Sie können in einem Prozess der schwachen Wechselwirkung in ein
Elektron und Neutrinos zerfallen und haben dabei eine mittlere Lebensdauer
$\tau = \qty{2.197}{\micro\s}$\cite{Workman:2022ynf}.
Die mittlere Lebensdauer eines Teilchens ist eine Kennziffer für
allgemeine Zerfallsprozesse, die in der Form
\begin{align}
	N(t) = N_0 e^{-\lambda t}
\end{align}
ablaufen wobei $\tau = \frac{1}{\lambda}$.
Myonen entstehen durch kosmische Strahlung in der oberen Atmosphäre.
und erreichen die Erdoberfläche aufgrund von der Zeitdilatation in
der speziellen Relativitätstheorie.% Rechnung nachreichen

Bei einer Ruhemasse des Myons von $m_\mu c² = \qty{105.7}{\MeV}$
einer mittleren Lebensdauer von $\tau = \qty{2.197e-6}{\s}$ (Vgl. \cite{Workman:2022ynf})
und einer Geschwindigkeit von $v \simeq c$ mit $c \simeq \qty{3e8}{\meter\per\second}$ ergibt sich eine mittlere
Reichweite von $R_\text{klassisch} = v\cdot\tau = c\cdot \tau \simeq \qty{660}{\m}$. 
Bei der Entstehung oberhalb von $\qty{10}{\km}$ ist also in der klassischen Rechnung  nicht mit Myonen auf der Erdoberfläche
zu rechnen.
Die relativistische Berechnung ergibt $\gamma = \frac{\qty{10}{\GeV}}{\qty{105.7}{\MeV}} \simeq \num{94.6}$
Im Ruhesystem ist die relativistische Lebensdauer dann $\tau' = \gamma \tau = 94.6 \cdot \qty{2.2e-6}{\s}=\qty{2.08e-4}{s} $
Es ergibt sich eine Reichweite von $R_\text{relativistisch}= v⋅τ'=c⋅τ'=\qty{3e8}{\meter\per\second} \cdot \qty{2.08e-4}{s}=\qty{62.4}{km}$

% Standardmodell
% Lebensdauer
% Reichweite klassisch und relativistisch (E_\mu = \qty{10}{\GeV})

\subsection{Messung von Szintillationsleuchten mit Photodetektoren \cite{book:kolano}}

Szintillatoren sind Materialien, die leuchten wenn sie von geladenen Teilchen durchquert werden.
Diese Teilchen Ionisieren die Atome im Szintillationsmedium. Bei der Rückkehr der Teilchen in den
Grundzustand geben diese Energie in Form von Licht ab. Der Szintillator ist so gewählt, dass das
Material durchsichtig für die Wellenlänge des erzeugten Lichtes ist.
Der Szintillatortank ist von einer reflektierenden Schicht umgeben, die dafür sorgt, dass das licht in
einem der beiden Photodetektoren landet.
Photodetektoren sind Messgräte, die dazu gedacht sind einzelne Photonen nachzuweisen.
Sie basieren auf dem Photoeffekt. Ein Photon löst ein Elektron auf einem unter hochspannung stehenden Kondensator aus.
Das Elektron wird zum Kondensator auf der nächsthöheren Spannungsstufe beschleunigt. Dort löst es weitere Elektronen
aus der nächsten Kondensatorplatte heraus. Diese Kaskade verstärkt sich über mehrere hintereinander geschaltete Kondensatorplatten
bis ein Messbares Signal entsteht.
Um das Hintergrundrauschen des Photodetektors von den tatsächlichen Signalen zu unterscheiden wird das Signal von zwei Photodetektoren
miteinander verglichen.

% Szintillator DONE
% Photodetektor DONE

\subsection{Fehlerrechnung}
Für die Fehlerrechnung werden alle \textbf{Mittelwerte} von $N$ Messungen folgendermaßen berechnet:

\begin{equation}
	\overline{x} = \frac{1}{N} \cdot \sum_{i=1}^N x_i
	\label{eqn:Mittelwert}
\end{equation}

und alle \textbf{Standardabweichungen zum Mittelwert} mit:

\begin{equation}
	\increment\overline{x} = \sqrt{\frac{1}{N\cdot(N-1)}\cdot\sum_{i=1}^N (x_i-\overline{x})^2}
	\label{eqn:St_Mittelwert}
\end{equation}

Der Fehler für zusammenhängende Messwerte wird dann mit der \textbf{Gaußschen Fehlerfortpflanzung} berechnet:

\begin{equation}
	\increment{f} = \sqrt{ \sum_{i = 1}^{N}  \biggl(\frac{\partial{f}}{\partial{x_i}}\biggr)^2\cdot(\increment{x_i})^2}
	\label{eqn:Gauss}
\end{equation}

Die Fehlerfortpflanzung wird mit Uncertainties in Python \cite{uncertainties} ermittelt.

%---------------------------------------------------------------------------------------------------------------------------------------------------------------%

\section{Durchführung \cite{man}}
% Diskriminator mit variabler Schwelle

\begin{figure}
	\centering
	\includegraphics[width=0.8\textwidth]{./Bilder/aufbau.png}
	\caption{Blockschaltbild des Versuchsaufbaus}\label{fig:aufbau}
\end{figure}

In Abbildung \ref{fig:aufbau} ist der Aufbau abgebildet.
Der Szintillator Tank mit den zwei PMT Detektoren ist über zwei Verzögerungsleitungnen
und zwei Diskriminatoren an eine Koinzidenzschaltung angeschlossen. 
Die Diskriminatoren filtern die Signale nach der Stärke, die Koinzidenzschaltung
lässt die Signale nur durch wenn sie von beiden PMT gleichzeitig kommt.
Signale die von den PMT gleichzeitig kommen sind durch ein geladenes Teilchen im Szintillator
ausgelöst, das beide Detektoren gleichzeitig aktivieren kann und nicht durch eine zufällige 
thermische Kaskade in einem der PMT.

Der zweite Teil der Schaltung ist dazu da die Signale von den Myonen herauszufiltern, die tatsächlich
in der Myonenkammer zerfallen


%% Monoflop Zeug Signal Kommm

%---------------------------------------------------------------------------------------------------------------------------------------------------------------%
\newpage
\section{Auswertung}
\subsection{Energiekalibrierung}
Um den Detektor zu kalibrieren, müssen zuerst die Peaks im Spektrum gefunden
werden, um diese dann mit der Literatur zu vergleichen. Dafür wird aus dem
Spektrum der Euridium Quelle

\begin{figure}[H]
	\centering
	\includegraphics[width=\textwidth]{build/plt1_Eu.pdf}
	\caption{Aufgenommenes Spektrum 152Eu}\label{fig:Eu_spektrum}
\end{figure}

die Peaks mittels $find-peaks$ von $scipy$ \cite{scipy} ermittelt. Jeder
Zählrate wird einem Channel zugeordnet, dessen Energie jedoch unbekannt ist.
Bei der Kalibrierung des Detektors wird die Position der charakteristischen
Peaks mit der aus Literatur \cite{web:Eu} bestimmten Energie des Peaks
verbunden. Dadurch lassen sich die Peaks in einem Energie zu Channel Diagramm
\eqref{fig:Eu_Fit} eintragen. Der sich ergebene linearer Zusammenhang wird
durch eine Ausgleichsgerade der Form

\begin{equation}
	E=m \cdot x +b
\end{equation}

bestimmt.

\begin{figure}[H]
	\centering
	\includegraphics[width=\textwidth]{build/plt2_Fit.pdf}
	\caption{Channel Energie Beziehung}\label{fig:Eu_Fit}
\end{figure}

Die Parameter dieser Ausgleichsgerade sind:

\begin{align*}
	m & =\qty[per-mode=fraction]{0.2073(0.0001)}{\kilo\eV\per\channel} \\
	b & =\qty{-1.026(0.327)}{\kilo\eV}
\end{align*}

Damit ist die Energiekalibrierung des Detektors abgeschlossen und jedem Channel
kann linear ein Energiewert zugeordnet werden.

\subsection{Vollenergienachweiswahrscheinlichkeit $Q$}
\subsubsection{Linieninhalt $Z$}
Um die Vollenergienachweiswahrscheinlichkeit $Q$ des Detektors zu bestimmen,
muss zuerst der Linieninhalt $Z$ der Einzellen Peaks bestimmt werden. Dazu wird
eine Gaußkurve

\begin{equation}
	g(x)=h\cdot \exp(-\frac{(x-\mu )^2}{2\sigma^2})+g
	\label{eq:Gauß}
\end{equation}

an jeden Einzellen Peak mittels $curve-fit$ von $scipy$ \cite{scipy} gefittet.
Dabei stellt $h$ die Höhe, $\mu$ den Mittelwert, $\sigma$ die
Standardabweichung und $g$ den störenden Hintergrund des Peaks dar. Beim Fitten
wurde die statistische Abweichung $\sqrt{N}$ der Zahlrate $N$ berücksichtigt,
sowie eine geschätzte Breite eines Peaks von $25$ Channel in jede Richtung
angenommen. Die verschiedenen Gaußkurven für die 8 unterschiedlichen Peaks sind
in Abbildung \eqref{fig:Gauß} gezeigt. Für alle Zukünftigen Gaußfits in der
Auswertung werden diese nicht immer explizit geplottet.

\begin{figure}
	\centering
	\includegraphics[width=\textwidth]{build/plt3_Gauß.pdf}
	\caption{Gaußfits der Peaks im Eu-Spektrum}
	\label{fig:Gauß}
\end{figure}

Um den Linieninhalt der Peaks schließlich zu berechnen wird Gaußkurve
integriert und die Fläche mittels

\begin{equation}
	Z=\sqrt{2\pi}\cdot h\sigma
	\label{eq:Z}
\end{equation}

bestimmt. Die Ergebnisse davon sind in der Tabelle ... zu finden, nachdem im
folgenden Abschnitt die Vollenergienachweiswahrscheinlichkeit berechnet wird.

\newpage
\subsubsection{Berechnung von $Q$}
Für die Berechnung dieser Vollenergienachweiswahrscheinlichkeit $Q$ wird Formel
... nach

\begin{equation}
	Q=\frac{4\pi \cdot Z}{\Omega \cdot AWT}
\end{equation}

umgestellt. Der Raumwinkel $\frac{\Omega}{4\pi}=0.0167 $ wird über der Formel
\eqref{eq:raumwinkel} mit $r=\qty{22.5}{\milli\meter}$ und
$d=\qty{85}{\milli\meter}$ berechnet. Die aktuelle Aktivität der Eu-Probe muss
aus der Ausgangsaktivität $A_0=\qty{4130(60)}{\becquerel}$ bei der Herstellung
am (01.10.2000) \cite{man:v18} berechnet werden. Dazu wird das Zerfallsgesetz

\begin{equation}
	A(t)=A_0 \cdot exp(-\frac{\ln(2)}{\tau }\cdot t)
\end{equation}

mit $\tau=\qty{13.5}{\year} $ benutzt. Nach mehr als 23 Jahren ergibt sich
damit eine Aktivität von

\begin{equation}
	A=\qty{1232(18)}{\becquerel}
\end{equation}

Zusammen mit den Emissionswahrscheinlichkeiten aus der Literatur \cite{web:Eu}
und einer Messzeit von $T=\qty{3413}{\second}$ ergeben sich folgende Werte:

\begin{table}[H]
	\centering
	\caption{Ergebnisse Vollenergienachweiswahrscheinlichkeit}
	\begin{tabular}{c c c c c}
		\toprule
		\text{Channel} & $ E [\unit{\kilo\eV}] $ & $ Z $               & $ W [\%] $ & $ Q [\%] $         \\
		\midrule
		594            & \num{123.2(0.1)}        & \num{9202.0(261.5)} & \num{28.6} & \num{46.00(01.47)} \\
		1186           & \num{245.9(0.1)}        & \num{1522.4(69.0)}  & \num{7.6}  & \num{28.68(01.36)} \\
		1666           & \num{345.4(0.2)}        & \num{3784.6(130.4)} & \num{26.5} & \num{20.40(00.76)} \\
		1985           & \num{411.6(0.2)}        & \num{317.4(40.4)}   & \num{2.2}  & \num{20.30(02.60)} \\
		2146           & \num{445.0(0.2)}        & \num{362.7(44.4)}   & \num{2.8}  & \num{18.37(02.26)} \\
		3760           & \num{779.6(0.4)}        & \num{718.2(43.7)}   & \num{12.9} & \num{07.93(00.50)} \\
		4190           & \num{868.8(0.5)}        & \num{270.3(35.1)}   & \num{4.2}  & \num{09.10(01.19)} \\
		4653           & \num{964.8(0.5)}        & \num{694.9(61.8)}   & \num{14.7} & \num{06.80(00.61)} \\
		\bottomrule
	\end{tabular}
	\label{tab:data1}
\end{table}

Da die Vollenergienachweiswahrscheinlichkeit nicht konstant ist, sondern von
der Energie abhängt, wird diese Abhängigkeit \eqref{fig:Eu_Q} geplottet und
eine Potenzfunktion der Form

\begin{equation}
	p(x)=a \cdot (x-b)^c
\end{equation}

gefittet.

\begin{figure}[H]
	\centering
	\includegraphics[width=\textwidth]{build/plt4_Q.pdf}
	\caption{Q in Abhängigkeit der Energie}
	\label{fig:Eu_Q}
\end{figure}

Die Werte dieser Parameter ergeben

\begin{align*}
	a & =\num{349803(723940)}         \\
	b & =\qty{-206.7(97.1)}{\kilo\eV} \\
	c & =\num{-1.54(0.28)}
\end{align*}

und werden im Laufe dieser Auswertung noch weiter verwendete werden.

\subsection{Untersuchung von 137Cs Gamma-Spektrums}
\subsubsection{Vollenergiepeak}
Das gemessene Gamma-Spektrum zusammen mit den gefunden Peaks ist in Abbildung
\eqref{fig:Cs_spektrum} zu sehen. Die Channels wurden mittels zuvor bestimmten
Werten direkt zu der jeweiligen Energie umgerechnet. Sowie sind die Peaks
wieder mittels $find-peaks$ von $scipy$ \cite{scipy} ermittelt worden. Es lässt
sich im Spektrum gut die Comptonkante (Peak $3$), den Rückstrahlpeak (Peak $2$)
sowie das Comptonkontinuum erkennen. Des Weiteren sticht der Vollenergiepeak
(Peak $4$) heraus.

\begin{figure}[H]
	\centering
	\includegraphics[width=\textwidth]{build/plt5_Cs.pdf}
	\caption{Aufgenommenes Spektrum 137Cs}
	\label{fig:Cs_spektrum}
\end{figure}

Dieser wird nun weiter untersucht. Dazu wird wieder ein Gaußkurve gefittet
\eqref{fig:Cs_peak} um den Inhalt des Peaks zu bestimmen. Außerdem wird der
Halbwertsbreite und die Zehntelwertsbreite des Peaks durch Umstellen der
Gaußkurve \eqref{eq:Gauß} nach $(x-\mu)$ über

\begin{align*}
	\frac{1}{2}g(\mu)  & \stackrel{!}{=}g(x_{1/2})  &  & \Rightarrow & E_{1/2}  & =2(x_{1/2}-\mu)=2\sigma \sqrt{2 \ln{\frac{2h}{h-g}}}    \\
	\frac{1}{10}g(\mu) & \stackrel{!}{=}g(x_{1/10}) &  & \Rightarrow & E_{1/10} & =2(x_{1/10}-\mu)=2\sigma \sqrt{2 \ln{\frac{10h}{h-9g}}}
\end{align*}

ermittelt. Der Inhalt des Vollenergiepeaks wird wieder über die Formel
\eqref{eq:Z} berechnet.

\begin{figure}[H]
	\centering
	\includegraphics[width=1\textwidth]{build/plt6_Ph_peak.pdf}
	\caption{Ausgeglichenes Absorptionsspektrum}
	\label{fig:Cs_peak}
\end{figure}

Insgesamt ergeben sich folgende Ergebnisse:

\begin{align*}
	E_{1/2}  & =\qty{2.21(0.05)}{\kilo\eV}   \\
	E_{1/10} & =\qty{4.02(0.09)}{\kilo\eV}   \\
	E_{max}  & =\qty{661.45(0.02)}{\kilo\eV} \\
	Z        & =\num{1.28(0.04)e4}           \\
\end{align*}

\subsubsection{Comptonkontinuum}
Wie bereits erwähnt lässt sich der Rückstrahlpeak und die Comptonkante gut
erkennen, auch wenn diese verschwommen sind. In der Abbildung
\eqref{fig:Compton} wird das Comptonkontinuum näher dargestellt. Die
theoretischen Positionen vom Rückstrahlpeak $E_{RP}$ und von der Comptonkante
$E_{CK}$ werden als senkrechte Linien dargestellt. Diese Werte werden über die
Formeln ... und ... mit $E_{\gamma}=\qty{661.45}{\kilo\eV}$ berechnet und
ergeben $E_{RP}=\qty{184.4}{\kilo\eV}$ und $E_{CK}=\qty{478.1}{\kilo\eV}$.

\begin{figure}[H]
	\centering
	\includegraphics[width=0.9\textwidth]{build/plt7_Compton.pdf}
	\caption{Ausgeglichenes Absorptionsspektrum}
	\label{fig:Compton}
\end{figure}

Das Comptonkontinuum wurde mittels der Klein-Nishina \eqref{eq:compton_energie} Formel gefittet.
Die Naturkonstanten wurden als konstanter Skalierungsfaktor betrachtet, um einen besseren Fit zu erreichen. 
Dabei wurden nur die Werte hinter der Comptonkante bis kurz vor dem Rückstrahlpeak genutzt, 
da der Bereich des Rückstrahlpeak den Fit verfälscht. 
Um den Linieninhalt zu berechnen wird mit den ermittelten werden die Klein-Nishina Formel integriert. 
Dazu wird von $scipy.integrate$ \cite{scipy} die Funktion $quad$ benutzt um das Integral zu approximieren. 
Damit ergibt sich ein Linieninhalt von:

\begin{equation*}
	Z_{Compton}=\num{31.091(0.176)e3}
\end{equation*}

\subsubsection{Absorptionswahrscheinlichkeit}
Aus der Detektorlänge $l=\qty{39}{\milli\meter}$ und den aus der Literatur ... entnommenen Extinktionskoeffizienten für den Photopeak und die Comptonkante lässen sich über 

\begin{equation}
	P=1-\exp(-\mu d)
\end{equation}

die Absorptionswahrscheinlichkeit $P_{Ph}=\num{1}$ und $P_{Ck}=\num{1}$ berechnen.

\subsection{Aktivitätsbestimmung}
Die Vermessung der Barium Quelle über ein Zeitraum von $T=\qty{3816}{\second}$ ergab das in Abbildung \eqref{fig:Ba_spektrum} zu erkennen Spektrum. 
Um die Aktivität der Quelle zu bestimmen, müssen die Linieninhalte der Peaks bestimmt werden um danach mittels der Formel \eqref{eq:Q} diese berechnen zu können. 
Der Linieninhalt wird wie zuvor mit einem Gaußfits bestimmt. 
Dessen Ergebnisse zusammen mit der aus der Literatur \cite{web:nuclear} bestimmten Emissionswahrscheinlichkeiten und 
der daraus berechneten Aktivität ist der Tabelle \eqref{tab:data2} zu entnehmen. 

\begin{figure}[H]
	\centering
	\includegraphics[width=0.95\textwidth]{build/plt8_Ba.pdf}
	\caption{Aufgenommenes Spektrum 133Ba}
	\label{fig:Ba_spektrum}
\end{figure}

\begin{table}[H]
	\centering
	\caption{Ergebnisse Vollenergienachweiswahrscheinlichkeit}
	\begin{tabular}{c c c c c}
		\toprule
		\text{Peak} & $ E [\unit{\kilo\eV}] $ & $ Z_{Ba} $               & $ W [\%] $ & $ A [\unit{\becquerel}] $         \\
		\midrule
		1           & \num{81.50}   &  \num{6850.29(382.71)}  & \num{34.06} & \num{560.64(31.32)} \\
		2           & \num{276.40}  &  \num{1020.26(56.61)}   & \num{7.16}  & \num{880.27(48.84)} \\
		3           & \num{302.94}  &  \num{2357.77(87.62)}   & \num{18.33} & \num{863.39(32.09)} \\
		4           & \num{356.02}  &  \num{6590.89(206.07)}  & \num{62.05} & \num{830.62(25.97)} \\
		5           & \num{384.01}  &  \num{871.814(52.94)}   & \num{8.94}  & \num{821.84(49.90)} \\
		\bottomrule
	\end{tabular}
	\label{tab:data2}
\end{table}

Im mittel ergibt sich eine Aktivität von $\qty{791(17)}{\becquerel}$.

\subsection{Gammaspektroskopie einer unbekannten Probe}
Um die Unbekannte Probe zu bestimmen, wurde diese mit dem Germanium-Detektor eine Stunde lang vermessen und die Peaks wieder mit $find-peaks$ ermittelt. 

\begin{figure}[H]
	\centering
	\includegraphics[width=\textwidth]{build/plt9_Un.pdf}
	\caption{Unbekanntes Spektrum}
	\label{fig:Un_spektrum}
\end{figure}

In der obigen Abbildung ist das Gamma-Spektrum sowie die Einzellen Peaks in Abhängigkeit der Energie gezeigt. 
Zur Identifizierung der Probe soll die Energie jedes Peak mit der Datenbank ... abgeglichen werden um die Wahrscheinlichste Quelle zu ermittelten. 
Dazu wurde um jeden Peak wieder ein Gaußfit gelegt um einen genaueren Wert für die Energie $E=\mu$ des Peaks zuhaben. 
Als Beispiel wird diese einmal mit dem größten Peak bei $E=\qty{295.36(0.01)}{\kilo\eV}$ mit $N=3160$ gemacht. 


%---------------------------------------------------------------------------------------------------------------------------------------------------------------%

\section{Diskussion}

%---------------------------------------------------------------------------------------------------------------------------------------------------------------%
\newpage
\printbibliography

\end{document}