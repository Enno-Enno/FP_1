\input{../header.tex}

\title{V18:\\ Germanium Detektor}
\author{Benedikt Lütke Lanfer \and Enno Wellmann}
\date{06. Mai 2024}
\publishers{TU Dortmund – Fakultät Physik}

\begin{document}
\begin{titlingpage}
    \begin{center}
        \begin{Huge}
            \textbf{\thetitle\\}
        \end{Huge}
    \end{center}
    \vspace{4cm}
    \includegraphics[width=\textwidth]{Bilder/Logo_TU.png} \\
    \vspace{4cm}
    \begin{center}
        \begin{huge}
            \theauthor\\
        \end{huge}
        \vspace{0.5cm}
        \begin{Large}
            benedikt.luetkelanfer@tu-dortmund.de\\
            enno.wellmann@tu-dortmund.de\\
            \vspace{1.4cm}
            Bearbeitet: \today\\
            Abgabe: \thedate\\
            TU Dortmund – Fakultät Physik\\
        \end{Large}
    \end{center}
\end{titlingpage}
\tableofcontents
\newpage

\section{Zielsetzung}
In diesem Versuch wird ein Germanium Energiedetektor mit einer geeichten \textbf{Eu} Quelle kalibriert.
Die

Stichworte:
\begin{itemize}
    \item Vollenergienachweiswahrscheinlichkeit
    \item Linien des Eu Spektrums 
    \item Halbwertsbreite
    \item Compton Kante
    \item 
\end{itemize}



%---------------------------------------------------------------------------------------------------------------------------------------------------------------%

\section{Theorie}
\subsection[]{Wechselwirkung von Strahlung mit Materie}
Aus \cite{book:knoll}.
Die hauptsächlichen Wechselwirkungen von Photonen mit Materie die photoelektrische Absorption,
die Compton Streuung und die paar Bildung.
Bei der photoelektrischen  Absorption wechselwirkt das Photon mit einem im Atom gebundenen Elektron und befördert es aus seinem gebundenen Zustand.
Die Energie des Photons wird danach von dem Elektron getragen und das Atom emittiert Röntgenstrahlung.
Diese Interaktion findet vor allem bei niederenergetischen $\gamma$ Photonen statt. 
Eine grobe Annäherung für das Verhalten der Wechselwirkungswahrscheinlichkeit
ist der Therm 
\begin{align}
    \tau = \text{constant} \cdot \frac{Z^n}{E_{\gamma}^{3.5}}
\end{align} %TODO herausfinden, was die Bedeutung von \tau ist.
wobei $n$ zwischen 4 und 5 liegt.

Die Compton Streuung ist oft die häufigste Interaktion der Strahlung mit dem Material.
Sie ist (abhängig von $Z$) dominant bei Energien im Bereich von etwa \qtyrange{0.5}{10}{\MeV}
Ihr Differentieller Wirkungsquerschnitt lautet
\begin{align}
    \frac{d\sigma}{d \Omega} = Z r_0^2 \left(\frac{1}{1+\alpha(1-\cos\theta)}\right)^2 %
    \left(\frac{1+ \cos^2\theta}{2}\right)%
    \left(1+ \frac{\alpha^2(1-\cos \theta)^2}{(1+\cos^2 \theta)[1+\alpha(1-\cos(\theta))]}\right)
    \label{eq:wq_compton}
\end{align}
wobei $\alpha = h \nu / m_0 c²$ und $r_0$ der klassische Elektronenradius ist (vgl. \cite{book:knoll}).

Bei Energien oberhalb der doppelten Ruhemasse des Elektrons (\qty{1.02}{\MeV}) ist die Elektronenpaarbildung möglich.
Diese Wechselwirkung tritt aber realistischerweise nur bei sehr hochenergetischen Gammastrahlen auf.
%Die Dunkelheit ist die Ursache des Lichts.

\subsection{Wahrscheinlichste Photonenenergien}
\begin{itemize} 
    \item[{^{152}Eu}] - \qty{121.7817} {\keV} \qty{28.58}{\%} 
    \item[{^{152}Eu}] - \qty{344.2785} {\keV} \qty{26.5}{\%}
    \item[{^{133}Ba}] - \qty{31}{\keV} peak
    \item[{^{133}Ba}] - \qty{356}{\keV} Breiterer peak
    \item[{^{125}Sb}] - \qty{427.88}{\keV} \qty{29.6}{\%}
\end{itemize}

\subsection{Fehlerrechnung}
Für die Fehlerrechnung werden alle \textbf{Mittelwerte} von $N$ Messungen folgendermaßen berechnet:

\begin{equation}
    \overline{x} = \frac{1}{N} \cdot \sum_{i=1}^N x_i
    \label{eqn:Mittelwert}
\end{equation}

und alle \textbf{Standardabweichungen zum Mittelwert} mit:

\begin{equation}
    \increment\overline{x} = \sqrt{\frac{1}{N\cdot(N-1)}\cdot\sum_{i=1}^N (x_i-\overline{x})^2}
    \label{eqn:St_Mittelwert}
\end{equation}

Der Fehler für zusammenhängende Messwerte wird dann mit der \textbf{Gaußschen Fehlerfortpflanzung} berechnet:

\begin{equation}
    \increment{f} = \sqrt{ \sum_{i = 1}^{N}  \biggl(\frac{\partial{f}}{\partial{x_i}}\biggr)^2\cdot(\increment{x_i})^2}
    \label{eqn:Gauss}
\end{equation}

Die Fehlerfortpflanzung wird mit Uncertainties in Python \cite{uncertainties} ermittelt.

%---------------------------------------------------------------------------------------------------------------------------------------------------------------%

\section{Durchführung}


%---------------------------------------------------------------------------------------------------------------------------------------------------------------%

\section{Auswertung}


%---------------------------------------------------------------------------------------------------------------------------------------------------------------%

\section{Diskussion}


%---------------------------------------------------------------------------------------------------------------------------------------------------------------%
\newpage
\printbibliography

\end{document}