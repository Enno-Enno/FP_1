\input{../header.tex}

\title{Vxxx:\\ Doffes Osziloskop}
\author{Benedikt Lütke Lanfer \and Enno Wellmann}
\date{13. Dezember 2024}
\publishers{TU Dortmund – Fakultät Physik}

\begin{document}
\begin{titlingpage}
    \begin{center}
        \begin{Huge}
            \textbf{\thetitle\\}
        \end{Huge}
    \end{center}
    \vspace{4cm}
    \includegraphics[width=\textwidth]{Bilder/Logo_TU.png} \\
    \vspace{4cm}
    \begin{center}
        \begin{huge}
            \theauthor\\
        \end{huge}
        \vspace{0.5cm}
        \begin{Large}
            benedikt.luetkelanfer@tu-dortmund.de\\
            enno.wellmann@tu-dortmund.de\\
            \vspace{1.4cm}
            Bearbeitet: \today\\
            Abgabe: \thedate\\
            TU Dortmund – Fakultät Physik\\
        \end{Large}
    \end{center}
\end{titlingpage}
\tableofcontents
\newpage

\section{Zielsetzung}

%---------------------------------------------------------------------------------------------------------------------------------------------------------------%

\section{Theorie}

% \subsection{Fehlerrechnung}
% Für die Fehlerrechnung werden alle \textbf{Mittelwerte} von $N$ Messungen folgendermaßen berechnet:

% \begin{equation}
%     \overline{x} = \frac{1}{N} \cdot \sum_{i=1}^N x_i
%     \label{eqn:Mittelwert}
% \end{equation}

% und alle \textbf{Standardabweichungen zum Mittelwert} mit:

% \begin{equation}
%     \increment\overline{x} = \sqrt{\frac{1}{N\cdot(N-1)}\cdot\sum_{i=1}^N (x_i-\overline{x})^2}
%     \label{eqn:St_Mittelwert}
% \end{equation}

% Der Fehler für zusammenhängende Messwerte wird dann mit der \textbf{Gaußschen Fehlerfortpflanzung} berechnet:

% \begin{equation}
%     \increment{f} = \sqrt{ \sum_{i = 1}^{N}  \biggl(\frac{\partial{f}}{\partial{x_i}}\biggr)^2\cdot(\increment{x_i})^2}
%     \label{eqn:Gauss}
% \end{equation}

% Die Fehlerfortpflanzung wird mit Uncertanties in Phython \cite{uncertainties} ermittelt.

%---------------------------------------------------------------------------------------------------------------------------------------------------------------%

\section{Durchführung}


%---------------------------------------------------------------------------------------------------------------------------------------------------------------%

\section{Auswertung}

\subsection{Schwellwert für die LASER Bildung}
In Abbildung \textbf{Abbildung einfügen} ist zu sehen wie der Laser auf die Karte strahlt.
Das für Menschen unsichtbare Licht mit einer Wellenlänge von 780 nm kann mit der Kamera eingefangen werden. %Vielleicht nicht ganz der richtige Ort für diese Anmerkung
Bei einem Strom unterhalb eines Schwellwerts von \textbf{WERTE einfügen} ist ein schwacher Punkt auf der Karte zu erkennen.
Oberhalb dieses Schwellwerts leuchtet dieser Punkt erheblich stärker und bildet kleine Leuchtflecken Rund um den Leuchtpunkt.

\subsection{Rubidium Fluoreszenz}
Bei einer Temperatur des Rb Gases von 50 °C kann mit der Kamera die Fluoreszenz beobachtet werden.
In Abbildung \textbf{einfügen} ist das Leuchten des Gases zu erkennen.
Entlang des Laserstrahls leuchtet das Gas verstärkt.


\subsection{Absorptionsspektrum}
Die 

%---------------------------------------------------------------------------------------------------------------------------------------------------------------%

\section{Diskussion}


%---------------------------------------------------------------------------------------------------------------------------------------------------------------%
\newpage
\printbibliography

\end{document}
