\input{../header.tex}

\title{V60:\\ Der Diodenlaser}
\author{Benedikt Lütke Lanfer \and Enno Wellmann}
\date{15. April 2024}
\publishers{TU Dortmund – Fakultät Physik}
\graphicspath{./Materialien}

\begin{document}
\begin{titlingpage}
    \begin{center}
        \begin{Huge}
            \textbf{\thetitle\\}
        \end{Huge}
    \end{center}
    \vspace{4cm}
    \includegraphics[width=\textwidth]{Bilder/Logo_TU.png} \\
    \vspace{4cm}
    \begin{center}
        \begin{huge}
            \theauthor\\
        \end{huge}
        \vspace{0.5cm}
        \begin{Large}
            benedikt.luetkelanfer@tu-dortmund.de\\
            enno.wellmann@tu-dortmund.de\\
            \vspace{1.4cm}
            Bearbeitet: \today\\
            Abgabe: \thedate\\
            TU Dortmund – Fakultät Physik\\
        \end{Large}
    \end{center}
\end{titlingpage}
\tableofcontents
\newpage

\section{Zielsetzung}
Ziel dieses Versuches ist ein Diodenlaser richtig zu kalibrieren, sodass damit das Absorptionsspektrum von Rubidium vermessen werden kann. 
%---------------------------------------------------------------------------------------------------------------------------------------------------------------%

\section{Theorie}
Die Informationen aus diesem Abschnitt wurden aus dem Handbuch zu diesem Versuch entnommen \cite{man:v60}
Vor der Erfindung des Diodenlasers waren die herkömmlichen Farblaser komplex und teuer. 
Außerdem waren sie schwierig zu bedienen, welches sich durch die Entwicklung der Verstellbaren Halbleiter Diodenlasers, mit enger Bandbreite, stark verbesserte. 

\subsection{Funktionsweise eines Diodenlasers}
Das prinzipielle Funktionsprinzip eines Diodenlasers unterscheidet sich am Anfang nicht von einer Halbleiterdiode. 
Bei dieser werden zwei Halbleiter mit n und p Dotierung aneinander gebracht, wodurch eine Grenzschicht oder auch Aktivschicht entsteht.  
Wenn nun ein Strom durch diese Grenzschicht fließt entstehen Elektronen Lochpaare, welche in dieser Aktivschicht sich wieder auslöschen. 
Diese geschieht jedoch erst ab einem bestimmten Schwellwert, bei dem die Elektronen Lochpaare entstehen. 
Bei diesem Prozess geben die Elektronen Licht mit der Wellenlänge ihrer Energie ab. 
Die Energie hängt wiederum fest von der Bandstruktur des Halbleiters ab.

\begin{wrapfigure}{r}{0.5\textwidth}
    \centering
    \vspace{-20pt}
    \includegraphics[width=0.4\textwidth]{Bilder/Bandstruktur.png} 
    \caption{Bsp: optischer Übergang in einer Bandstruktur \cite{Bandstruktur}}
    \label{fig:Bandstruktur}
\end{wrapfigure}

In der nebenstehenden Abbildung \eqref{fig:Bandstruktur} ist eine vereinfachte Bandstruktur beispielhaft abgebildet. 
Wenn nun Elektronen-Lochpaare sich gegenseitig eliminieren, so findet ein optischer Übergang vom Leitungsband ins Valenzband statt. 
Dieser hat eine feste Energie und somit hat auch das emittierte Licht eine nahezu feste Wellenlänge, 
welche in einem dünnen Channel im Chip \eqref{fig:Schemata} eingeschränkt ist. 
Die Enden dieses Chips verhalten sich, durch ihren hohen Brechungsindex, wie partiell reflektierende Spiegel, die das Licht teils einschließen. 
Wenn das Licht im Chip hin und her reflektiert wird, regt es angeregte Atome ab, welche wiederum Licht mit der exakt selben Wellenlänge und Richtung emittiert. 
Dieses wird stimulierte Emissionen genannt, welche dafür sorgen, dass Licht einer bestimmten Wellenlänge verstärkt wird.  

\begin{figure}[H]
    \centering
    \includegraphics[width=\textwidth]{Bilder/Schemata.png} 
    \caption{Aufbau des Diodenchips \cite{man:v60}}
    \label{fig:Schemata}
\end{figure}

Das Licht, welches dann aus dem Chip austritt, ist aufgrund der schmalen Austrittsfläche stark elliptisch und divergent, welches mit einer Linse korrigiert wird. 
Danach kommt ein Beugungsgitter, welches je nach Winkel eine bestimmte Wellenlänge des Lichtes wieder in den Laser zurückwirft. 
Dadurch wird im Laser die stimulierte Emission vermehrt mit dieser Wellenlänge durchgeführt und der LASER strahlt weniger anders welliges Licht aus.
Damit kann man den LASER genauer auf die gewünschte Wellenlänge einstellen, die gebraucht wird. 
Die genaue Feineinstellung des LASER wird im folgenden Abschnitt erklärt. 

\newpage
\subsection{Beiträge zur Gesamtintensität}
Die verschiedenen Anteile an der Gesamtintensität des LASER sind in folgender Abbildung abgebildet. 

\begin{figure}[H]
    \centering
    \includegraphics[width=0.9\textwidth]{Bilder/LASER_Moden.png} 
    \caption{Die verschiedenen Beiträge zur Gesamtintensität \cite{man:v60}}
    \label{fig:Moden}
\end{figure}

\begin{description}
    \item[1.Das Medium] des Halbleiterchips und dessen Bandlücke ist stark temperaturabhängig.
    Bei höheren Temperaturen wird die Bandlücke des Halbleiters kleiner und somit die Wellenlänge größer ($\qty{0.23}{\nano\meter\per\celsius}$). 
    Es wird Licht in einem breiten Frequenzband emittiert.
    \item[2.Die innere Cavity] oder auch optischer Hohlraumresonator sorgt dafür, dass sich ganzzahlige vielfache stehender Wellen im Chip bilden.
    Die Periodizität dessen nennt man "freien Spektralbereich" und beträgt für diesen 
    LASER $\Delta_{FSR} \approx  \qty{60}{\giga\hertz}$. 
    Auch hier ist die Wellenlänge Temperatur empfindlich aber hängt auch vom Strom ab. 
    Dieser kann sowohl die Temperatur wieder erhöhen als auch die Elektronen-Lochpaar-Konzentration ändern. 
    \item[3.Das Beugungsgitter Feedback] sorgt für besseres regulieren der gewünschten Wellenlänge mit stimulierte Emissionen. 
    Durch Ändern des Winkels $\theta $ des Gitters wird die richtig Wellenlänge, nach $\lambda =2d \sin\theta $ in den LASER zurückgeworfen. 
    \item[4.Die äußere Cavity] ist deutlich größer als die innere mit $\Delta_{FSR} \approx  \qty{10}{\giga\hertz}$ 
    und kann durch einen piezo-elektrischen Aktuator verändert werden. 
\end{description}

\subsection{Mode Hop}
Die verschiedenen Einstellungsmöglichkeiten beim LASER reagieren nicht alle gleich schnell oder linear auf Veränderungen.
Wenn die innere Cavity konstant gehalten wird und nur der Winkel des Gitters und die äußere Cavity verändert werden 
kann ein Mode Hop auftreten (vgl. Abbildung \ref{fig:Hop}).
Am Anfang seien die Intensitätsmaxima von inneren (Int0) und äußere Cavity sowie des Gitterfeedbacks(e0) überlagert.
Falls nun Veränderungen vorgenommen werden, 
bleibt das innere Maximum bei derselben Wellenlänge, während das andere Maximum sich verschiebt. 
Der LASER bleibt jedoch bei der bisherigen betriebenen Wellenlänge. 
Wenn das Maximum von der äußeren Cavity und das des Gitters sich weit genug verschoben haben, gibt es einen Mode Hop, 
da das zweite Maximum (Int1) der inneren Cavity mehr überlagert als das erste (Int0).
Dadurch macht die Frequenz des Lasers ein Sprung von etwa $\qty{20}{\giga\hertz}$.
Soll der Laser für eine Spektralanalyse verwendet werden sind mode hops störend, 
da sie eine Zuordnung der Piezo-Bewegung und der Laser Frequenz unmöglich machen.

\begin{figure}[H]
    \centering
    \includegraphics[width=\textwidth]{Bilder/Mode_Hop.png} 
    \caption{Grafische Darstellung eines Mode Hops \cite{man:v60}}
    \label{fig:Hop}
\end{figure}

Um die LASER ohne diese Mode Hop genauer einstellen zu können muss auch die innere Cavity, durch Veränderung des Stroms, gleichzeitig angepasst werden.
Dadurch kann man die gesamte Frequenzbreite des LASER abdecken und kalibrieren. 


\subsection{Fehlerrechnung}
Für die Fehlerrechnung werden alle \textbf{Mittelwerte} von $N$ Messungen folgendermaßen berechnet:

\begin{equation}
    \overline{x} = \frac{1}{N} \cdot \sum_{i=1}^N x_i
    \label{eqn:Mittelwert}
\end{equation}

und alle \textbf{Standardabweichungen zum Mittelwert} mit:

\begin{equation}
    \increment\overline{x} = \sqrt{\frac{1}{N\cdot(N-1)}\cdot\sum_{i=1}^N (x_i-\overline{x})^2}
    \label{eqn:St_Mittelwert}
\end{equation}

Der Fehler für zusammenhängende Messwerte wird dann mit der \textbf{Gaußschen Fehlerfortpflanzung} berechnet:

\begin{equation}
    \increment{f} = \sqrt{ \sum_{i = 1}^{N}  \biggl(\frac{\partial{f}}{\partial{x_i}}\biggr)^2\cdot(\increment{x_i})^2}
    \label{eqn:Gauss}
\end{equation}

Die Fehlerfortpflanzung wird mit Uncertainties in Python \cite{uncertainties} ermittelt.

%---------------------------------------------------------------------------------------------------------------------------------------------------------------%

\section{Durchführung}
Zuerst soll der Schwellwert des LASER bestimmt werden. 
Dazu wird langsam der Strom erhöht bis das Licht des LASER signifikant stärker wird. 
Danach wird der Laserstrahl durch Rubidiumgas geschickt um dessen Fluoreszenz mittels einer Kamera zu beobachten. 
Zum Schluss wird durch Feineinstellungen des LASER das Absorptionsspektrum von Rubidium vermessen. 
Dafür wird am LASER eine Dreieckspannung angelegt und dessen Strahl wieder durch das Rubidiumgas geschickt.
Am Ende der Apparatur wird die Intensität des Strahls von einer Fotodiode gemessen. 
Um ein Vergleichswert zu haben wird der Strahl vorher von einem halbdurchlässigen Spiegel in zwei geteilt. 
Die eine Hälfte geht durch das Rubidium die andere Hälfte auf eine Fotodiode und die Signale werden von einander abgezogen.


%---------------------------------------------------------------------------------------------------------------------------------------------------------------%

\newpage
\section{Auswertung}

\subsection{Schwellwert für die LASER Bildung}
In Abbildung \ref{fig:laser_karte} ist zu sehen wie der Laser auf die Karte strahlt.
Das für Menschen unsichtbare Licht mit einer Wellenlänge von 780 nm kann mit der Kamera eingefangen werden. %Vielleicht nicht ganz der richtige Ort für diese Anmerkung
Bei einem Strom unterhalb eines Schwellwerts ist ein schwacher Punkt auf der Karte zu erkennen.
Oberhalb davon leuchtet dieser Punkt erheblich stärker und bildet kleine Leuchtflecken rund um den Leuchtpunkt.
Der Wert wird ermittelt, indem sich schrittweise von beiden Seiten daran angenähert wird.
\begin{table}
    \centering
    \begin{tabular}[table-format=2.1]{S S}
        \toprule
        {Unter dem Schwellwert} & {Über dem Schwellwert} \\
        {$I/\si{\milli\ampere}$} & {$I/\si{\milli\ampere}$}\\
        \midrule
        34.8  & 36.0 \\
        34.6  & 34.9 \\
              & 34.8 \\
        \bottomrule    
    \end{tabular}
\end{table}

Es ergibt sich ein mittlere Schwellwert bei $I = \qty{34.97}{\milli\ampere}$.

\begin{figure}
    \centering
    \subfloat[Lichtpunkt im dioden Zustand des Lasers]{\includegraphics[width=0.4\textwidth]{./Materialien/karte_laser_schwach.jpeg}}
    \hfill
    \subfloat[Licht im Laser Zustand]{\includegraphics[width=0.4\textwidth]{./Materialien/karte_laser_stark.jpeg}}
    \caption{Zustände des Lasers}\label{fig:laser_karte}
\end{figure}

\subsection{Rubidium Fluoreszenz}
Bei einer Temperatur des Rb Gases von 50 °C kann mit der Kamera die Fluoreszenz beobachtet werden.
In Abbildung \ref{fig:rb_fluoreszenz} ist das Leuchten des Gases zu erkennen.
Entlang des Laserstrahls leuchtet das Gas verstärkt.
\begin{figure}
    \centering
    \subfloat[ohne Fluoreszenz]{\includegraphics[width=0.4\textwidth]{./Materialien/Rb_Zelle_keine_Floureszenz.jpeg}}
    \hfill
    \subfloat[mit Fluoreszenz]{\includegraphics[width=0.4\textwidth]{./Materialien/Rb_Zelle_Floureszenz.jpeg}}
    \caption{Rb Zelle mit und ohne sichtbarer Fluoreszenz.}\label{fig:rb_fluoreszenz}
\end{figure}

\subsection{Absorptionsspektrum}
Durch Manipulieren des DC-Offsets und des Beugungsgitters
lässt sich das Spektrum in einer kontinuierlichen Art und Weise darstellen.
Der Abgleich zwischen Referenzsignal und gemessenen Signal ergibt ein Absorptionsspektrum mit ebener Grundlinie.
Das entstehende Spektrum ist in Abbildung \ref{fig:spektrum} zu sehen.

\begin{figure}
    \centering
    \includegraphics[width=0.4\textwidth]{./Materialien/spektrum.jpeg}
	\caption{Ausgeglichenes Absorptionsspektrum}\label{fig:spektrum}
\end{figure}

%---------------------------------------------------------------------------------------------------------------------------------------------------------------%
\newpage
\section{Diskussion}
Der Diodenlaser wurde in diesem Experiment erfolgreich in Betrieb genommen.
Durch die Verwendung der Kamera und der sichtbar machenden Karte konnte der Weg des Laserstrahls nachvollzogen werden.
In der mit Rubidiumgas gefüllten Kammer konnte Fluoreszenz beobachtet werden und das Absorptionsspektrum in Abhängigkeit
der Wellenlänge gemessen werden.
Das durchgeführte Experiment ist eine funktionierende Grundlage von der ausgehend
 eine weitergehende Spektralanalyse durchgeführt werden könnte. 


%---------------------------------------------------------------------------------------------------------------------------------------------------------------%

\printbibliography

\end{document}
