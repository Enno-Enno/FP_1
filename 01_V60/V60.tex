\input{../header.tex}

\title{V60:\\ Der Diodenlaser}
\author{Benedikt Lütke Lanfer \and Enno Wellmann}
\date{15. April 2024}
\publishers{TU Dortmund – Fakultät Physik}

\begin{document}
\begin{titlingpage}
    \begin{center}
        \begin{Huge}
            \textbf{\thetitle\\}
        \end{Huge}
    \end{center}
    \vspace{4cm}
    \includegraphics[width=\textwidth]{Bilder/Logo_TU.png} \\
    \vspace{4cm}
    \begin{center}
        \begin{huge}
            \theauthor\\
        \end{huge}
        \vspace{0.5cm}
        \begin{Large}
            benedikt.luetkelanfer@tu-dortmund.de\\
            enno.wellmann@tu-dortmund.de\\
            \vspace{1.4cm}
            Bearbeitet: \today\\
            Abgabe: \thedate\\
            TU Dortmund – Fakultät Physik\\
        \end{Large}
    \end{center}
\end{titlingpage}
\tableofcontents
\newpage

\section{Zielsetzung}
Ziel dieses Versuches ist ein Diodenlaser richtig zu kalibrieren, sodass damit das Absorption Spectrum von Rubidium vermessen werden kann. 
%---------------------------------------------------------------------------------------------------------------------------------------------------------------%

\section{Theorie}
Bevor der Erfindung des Diodenlasers waren die herkömmlichen Farblaser komplex und teuer. 
Außerdem waren sie schwierig zu bedienen, welches sich durch die Entwicklung der Verstellbaren Halbleiter Diodenlasers, mit enger Bandbreite, stark verbesserte. 

\subsection{Funktionsweise eines Diodenlasers}
Das prinzipielle Funktionsprinzip eines Diodenlasers unterscheidet sich am Anfang nicht von einer Halbleiterdiode. 
Bei dieser werden zwei Halbleiter mit n und p Dotierung aneinander gebracht, wodurch eine Grenzschicht oder auch Aktivschicht entsteht.  
Wenn nun ein Strom durch diese Grenzschicht fließt entstehen Elektronen Lochpaare, welche in dieser Aktienschicht sich wieder auslöschen. 
Bei diesem Prozess geben die Elektronen Licht mit der Wellenlänge ihrer Energie ab. 
Die Energie hängt wiederum fest von der Bandstruktur des Halbleiters ab.

\begin{wrapfigure}{r}{0.5\textwidth}
    \centering
    \vspace{-20pt}
    \includegraphics[width=0.4\textwidth]{Bilder/Bandstruktur.png} 
    \caption{Bsp: optischer Übergang in einer Bandstruktur \cite{Bandstruktur}}
    \label{fig:Bandstruktur}
\end{wrapfigure}

In der nebenstehenden Abbildung \eqref{fig:Bandstruktur} ist eine vereinfachte Bandstruktur beispielhaft abgebildet. 
Wenn nun Elektronen-Lochpaare sich gegenseitig eliminieren, so findet ein optischer Übergang vom Leitungsband ins Valenzband statt. 
Dieser hat eine feste Energie und somit hat auch das emittierte Licht eine nahezu feste Wellenlänge, 
welche in einem dünnen Channel im Chip \eqref{fig:Schemata} eingeschränkt ist. 
Die Enden dieses Chips verhalten sich, durch ihren hohen Brechungsindex, wie partiell reflektierende Spiegel, die das Licht teils einschließen. 
Wenn das Licht im Chip hin und her reflektiert wird, regt es angeregte Atome ab, welche wiederum Licht mit der exakt selben Wellenlänge und Richtung emittiert. 
Dieses wird stimulierte Emissionen genannt, welche dafür sorgen, dass Licht einer bestimmten Wellenlänge verstärkt wird.  

\begin{figure}[H]
    \centering
    \includegraphics[width=\textwidth]{Bilder/Schemata.png} 
    \caption{Aufbau des Diodenchips \cite{bk:LASER}}
    \label{fig:Schemata}
\end{figure}

Das Licht, welches dann aus dem Chip austritt, ist aufgrund der schmalen Austrittsfläche stark elliptisch und divergent, welches mit einer Linse korrigiert wird. 
Danach kommt ein Beugungsgitter, welches je nach Winkel eine bestimmte Wellenlänge des Lichtes wieder in den Laser zurückwirft. 
Dadurch wird im Laser die stimulierte Emission vermehrt mit dieser Wellenlänge durchgeführt und der LASER strahlt weniger anders welliges Licht aus.
Damit kann man den LASER genauer auf die gewünschte Wellenlänge einstellen, die gebraucht wird. 
Die genaue Feineinstellung des LASER wird im folgenden Abschnitt erklärt. 

\newpage
\subsection{Beiträge zur Gesamtintensität}
Die verschieden Anteile an der Gesamtintenstiät des LASER sind in folgender Abbildung abgebildet. 

\begin{figure}[H]
    \centering
    \includegraphics[width=\textwidth]{Bilder/LASER_Moden.png} 
    \caption{Die verschiedenen Beiträge zur Gesamtintensität \cite{bk:LASER}}
    \label{fig:Schemata}
\end{figure}

\begin{description}
    \item[1.Medium] Das Material des Halbleiterchips und dessen Bandlücke ist stark Temperatur abhängig. 
    Dadurch wird bei höheren Temperaturen die Wellenlänger größer (\qty{0.23}{\nano\meter\per\celsius}). 
    \item[2.Innerer Cavity] Zweiter Punkt
    \item[3.Beugungsgitter Feedback] Dritter Punkt
    \item[4.Äußerer Cavity] Dritter Punkt
\end{description}





\subsection{Fehlerrechnung}
Für die Fehlerrechnung werden alle \textbf{Mittelwerte} von $N$ Messungen folgendermaßen berechnet:

\begin{equation}
    \overline{x} = \frac{1}{N} \cdot \sum_{i=1}^N x_i
    \label{eqn:Mittelwert}
\end{equation}

und alle \textbf{Standardabweichungen zum Mittelwert} mit:

\begin{equation}
    \increment\overline{x} = \sqrt{\frac{1}{N\cdot(N-1)}\cdot\sum_{i=1}^N (x_i-\overline{x})^2}
    \label{eqn:St_Mittelwert}
\end{equation}

Der Fehler für zusammenhängende Messwerte wird dann mit der \textbf{Gaußschen Fehlerfortpflanzung} berechnet:

\begin{equation}
    \increment{f} = \sqrt{ \sum_{i = 1}^{N}  \biggl(\frac{\partial{f}}{\partial{x_i}}\biggr)^2\cdot(\increment{x_i})^2}
    \label{eqn:Gauss}
\end{equation}

Die Fehlerfortpflanzung wird mit Uncertanties in Phython \cite{uncertainties} ermittelt.

%---------------------------------------------------------------------------------------------------------------------------------------------------------------%

\section{Durchführung}


%---------------------------------------------------------------------------------------------------------------------------------------------------------------%

\section{Auswertung}


%---------------------------------------------------------------------------------------------------------------------------------------------------------------%

\section{Diskussion}


%---------------------------------------------------------------------------------------------------------------------------------------------------------------%
\newpage
\printbibliography

\end{document}