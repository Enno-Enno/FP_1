\input{../header.tex}

\title{V01:\\ Kosmische Myonen}
\author{Benedikt Lütke Lanfer \and Enno Wellmann}
\date{01.Juli 2024}
\publishers{TU Dortmund – Fakultät Physik}

\begin{document}
\begin{titlingpage}
    \begin{center}
        \begin{Huge}
            \textbf{\thetitle\\}
        \end{Huge}
    \end{center}
    \vspace{4cm}
    \includegraphics[width=\textwidth]{Bilder/Logo_TU.png} \\
    \vspace{4cm}
    \begin{center}
        \begin{huge}
            \theauthor\\
        \end{huge}
        \vspace{0.5cm}
        \begin{Large}
            benedikt.luetkelanfer@tu-dortmund.de\\
            enno.wellmann@tu-dortmund.de\\
            \vspace{1.4cm}
            Bearbeitet: \today\\
            Abgabe: \thedate\\
            TU Dortmund – Fakultät Physik\\
        \end{Large}
    \end{center}
\end{titlingpage}
\tableofcontents
\newpage

\section{Zielsetzung}

%---------------------------------------------------------------------------------------------------------------------------------------------------------------%

\section{Theorie}
\subsection{Myonen}
Myonen sind Teilchen des Standardmodells.
Sie können in einem Prozess der schwachen Wechselwirkung in ein
Elektron und Neutrinos zerfallen und haben dabei eine mittlere Lebensdauer
$\tau = \qty{2.197}{\micro\s}$\cite{Workman:2022ynf}.
Die mittlere Lebensdauer eines Teilchens ist eine Kennziffer für
allgemeine Zerfallsprozesse, die in der Form
\begin{align}
	N(t) = N_0 e^{-\lambda t}
\end{align}
ablaufen wobei $\tau = \frac{1}{\lambda}$.
Myonen entstehen durch kosmische Strahlung in der oberen Atmosphäre.
und erreichen die Erdoberfläche aufgrund von der Zeitdilatation in
der speziellen Relativitätstheorie.% Rechnung nachreichen

Bei einer Ruhemasse des Myons von $m_\mu c² = \qty{105.7}{\MeV}$
einer mittleren Lebensdauer von $\tau = \qty{2.197e-6}{\s}$ (Vgl. \cite{Workman:2022ynf})
und einer Geschwindigkeit von $v \simeq c$ mit $c \simeq \qty{3e8}{\meter\per\second}$ ergibt sich eine mittlere
Reichweite von $R_\text{klassisch} = v\cdot\tau = c\cdot \tau \simeq \qty{660}{\m}$. 
Bei der Entstehung oberhalb von $\qty{10}{\km}$ ist also in der klassischen Rechnung  nicht mit Myonen auf der Erdoberfläche
zu rechnen.
Die relativistische Berechnung ergibt $\gamma = \frac{\qty{10}{\GeV}}{\qty{105.7}{\MeV}} \simeq \num{94.6}$
Im Ruhesystem ist die relativistische Lebensdauer dann $\tau' = \gamma \tau = 94.6 \cdot \qty{2.2e-6}{\s}=\qty{2.08e-4}{s} $
Es ergibt sich eine Reichweite von $R_\text{relativistisch}= v⋅τ'=c⋅τ'=\qty{3e8}{\meter\per\second} \cdot \qty{2.08e-4}{s}=\qty{62.4}{km}$

% Standardmodell
% Lebensdauer
% Reichweite klassisch und relativistisch (E_\mu = \qty{10}{\GeV})

\subsection{Messung von Szintillationsleuchten mit Photodetektoren \cite{book:kolano}}

Szintillatoren sind Materialien, die leuchten wenn sie von geladenen Teilchen durchquert werden.
Diese Teilchen Ionisieren die Atome im Szintillationsmedium. Bei der Rückkehr der Teilchen in den
Grundzustand geben diese Energie in Form von Licht ab. Der Szintillator ist so gewählt, dass das
Material durchsichtig für die Wellenlänge des erzeugten Lichtes ist.
Der Szintillatortank ist von einer reflektierenden Schicht umgeben, die dafür sorgt, dass das licht in
einem der beiden Photodetektoren landet.
Photodetektoren sind Messgräte, die dazu gedacht sind einzelne Photonen nachzuweisen.
Sie basieren auf dem Photoeffekt. Ein Photon löst ein Elektron auf einem unter hochspannung stehenden Kondensator aus.
Das Elektron wird zum Kondensator auf der nächsthöheren Spannungsstufe beschleunigt. Dort löst es weitere Elektronen
aus der nächsten Kondensatorplatte heraus. Diese Kaskade verstärkt sich über mehrere hintereinander geschaltete Kondensatorplatten
bis ein Messbares Signal entsteht.
Um das Hintergrundrauschen des Photodetektors von den tatsächlichen Signalen zu unterscheiden wird das Signal von zwei Photodetektoren
miteinander verglichen.

% Szintillator DONE
% Photodetektor DONE

\subsection{Fehlerrechnung}
Für die Fehlerrechnung werden alle \textbf{Mittelwerte} von $N$ Messungen folgendermaßen berechnet:

\begin{equation}
	\overline{x} = \frac{1}{N} \cdot \sum_{i=1}^N x_i
	\label{eqn:Mittelwert}
\end{equation}

und alle \textbf{Standardabweichungen zum Mittelwert} mit:

\begin{equation}
	\increment\overline{x} = \sqrt{\frac{1}{N\cdot(N-1)}\cdot\sum_{i=1}^N (x_i-\overline{x})^2}
	\label{eqn:St_Mittelwert}
\end{equation}

Der Fehler für zusammenhängende Messwerte wird dann mit der \textbf{Gaußschen Fehlerfortpflanzung} berechnet:

\begin{equation}
	\increment{f} = \sqrt{ \sum_{i = 1}^{N}  \biggl(\frac{\partial{f}}{\partial{x_i}}\biggr)^2\cdot(\increment{x_i})^2}
	\label{eqn:Gauss}
\end{equation}

Die Fehlerfortpflanzung wird mit Uncertainties in Python \cite{uncertainties} ermittelt.

%---------------------------------------------------------------------------------------------------------------------------------------------------------------%

\section{Durchführung \cite{man}}
% Diskriminator mit variabler Schwelle

\begin{figure}
	\centering
	\includegraphics[width=0.8\textwidth]{./Bilder/aufbau.png}
	\caption{Blockschaltbild des Versuchsaufbaus}\label{fig:aufbau}
\end{figure}

In Abbildung \ref{fig:aufbau} ist der Aufbau abgebildet.
Der Szintillator Tank mit den zwei PMT Detektoren ist über zwei Verzögerungsleitungnen
und zwei Diskriminatoren an eine Koinzidenzschaltung angeschlossen. 
Die Diskriminatoren filtern die Signale nach der Stärke, die Koinzidenzschaltung
lässt die Signale nur durch wenn sie von beiden PMT gleichzeitig kommt.
Signale die von den PMT gleichzeitig kommen sind durch ein geladenes Teilchen im Szintillator
ausgelöst, das beide Detektoren gleichzeitig aktivieren kann und nicht durch eine zufällige 
thermische Kaskade in einem der PMT.

Der zweite Teil der Schaltung ist dazu da die Signale von den Myonen herauszufiltern, die tatsächlich
in der Myonenkammer zerfallen


%% Monoflop Zeug Signal Kommm
%---------------------------------------------------------------------------------------------------------------------------------------------------------------%

\section{Auswertung}
\subsection{Anmerkungen}
Die aufgenommenen Daten zur Myonenlebenszeit geben nicht den erwarteten Verlauf wieder und besitzen eine viel zu hohe Zählrate, als von Myonen zu erwarten.
Dieses liegt an einem Fehler beim Kalibrieren der Apparatur und dem Messen. 
Einer der Diskriminatoren die benutzt wurden, ist defekt und konnte nicht eingestellt werden. 
Dieses ist bei der Kalibrierung der Diskriminatoren uns nicht aufgefallen, sondern erst am Ende der Messzeit. 
Es wurde zwar versucht eine neue Messreihe, mit einem anderen kalibrierten Diskriminator, aufzunehmen, 
jedoch stürzte der PC mitten in der Messreihe ab, sodass die Daten verloren gingen. 
Eine dritte Messreihe hat nicht genug Datenpunkte um, eine Analyse zu beginnen. 
Daher wird mit dem fehlerhaften Datensatz gearbeitet und eine ungefähre Bestimmung der Lebensdauer von Myonen versucht.    

\subsection{Bestimmung der Verzögerungszeit}
Die aufgenommenen Zählraten \eqref{tab:data1} wurden gegen die jeweilige Verzögerungszeit $T_{vz}$ geplottet \eqref{fig:plt1} 
und das sich abzeichnende Plateau über den Mittelwert bestimmt. Dieser ergibt ein Wert von $\num{159.5}$ im Bereich von $\qty{4}{\us}$ bis $\qty{14}{\us}$.
Für die folgenden Messungen wurde eine Verzögerung von $\qty{10}{\us}$ gewählt, welches mittig im Intervall liegt. 

\begin{figure}[H]
	\centering
	\includegraphics[width=0.9\textwidth]{build/plot1.pdf}
	\caption{Verzögerungszeit Bestimmung}\label{fig:plt1}
\end{figure}

\begin{table}[H]
	\centering
	\begin{tabular}{c c}
		\toprule
		$T_{vz} \, [\unit{\us}]$ & $N $  \\
		\midrule
        0  & 128 \\
        1  & 123 \\
        2  & 118 \\
        4  & 157 \\ 
        6  & 151 \\
        8  & 149 \\
        10 & 169 \\
        12 & 156 \\
        14 & 179 \\
        16 & 118 \\
        20 & 85  \\ 
        24 & 44  \\
        32 & 29  \\
		\bottomrule
	\end{tabular}
    \caption{Messdaten der Versicherungsmessung}
    \label{tab:data1}
\end{table}

\subsection{Kalibrierung}

\begin{wrapfigure}{r}{0.6\textwidth}
	\centering
	\includegraphics[width=0.6\textwidth]{build/plot2.pdf}
	\caption{Kalibrierungsdaten}\label{fig:plt2}
\end{wrapfigure}

Die Kalibrierung des Viel-Kanal-Analysator ergab das in die nebenstehende Abbildung \eqref{fig:plt2} geplottete Spektrum. 
Der Doppelimpulsgenerator wurde immer um $\qty{0.5}{\us}$, beginnend bei $\qty{0.4}{\us}$ erhöht, daher hat jeder Peak im Spektrum diesen Abstand.
Die Peaks werden mittels  $find-peaks$ von $scipy$ \cite{scipy} bestimmt und danach die Zeit $t$ gegen die Channels aufgetragen. 
Dabei wird eine Ausgleichsgerade der Form $f(x)=m \cdot x+b$ durch die Messwerte gelegt und dessen Steigung bestimmt. 


\begin{figure}[H]
	\centering
	\includegraphics[width=\textwidth]{build/plot3.pdf}
	\caption{Ausgleichsgerade Kalibrierung}\label{fig:plt3}
\end{figure}


Die Parameter der Ausgleichsgerade

\begin{align*}
	m&=\qty{21,67(0,01)e-3}{\us\per\channel}\\
	b&=\qty{0,1651(0,0026)}{\us}
\end{align*}

werden im Folgenden benutzt, um die Channels in Zerfallszeit umzurechnen. 

\subsection{Lebensdauerbestimmung}
Wie bereits erwähnt sind die in Abbildung \eqref{fig:plt3} zusehenden Daten fehlerhaft. 
Neben einer viel zu hohen Zahlrate von Insgesamt $1.648.692$ innerhalb von einer Messzeit von $T_{mess}=\qty{157226}{\s}$,
weist das Spektrum eine ungewöhnliche Kante bei $t_k=\qty{1.075}{\us}$ auf.
Daher kann angenommen das die Myonensignale, gerade für kleine Zeiten $t<t_k$, von anderen unbekannten Signalen überlagert werden. 
Daher ist eine theoretische Berechnung des Hintergrundes unpraktisch, sondern dieser wird in den jeweiligen Ausgleichsrechnungen berücksichtigt. 
Im Folgenden werden verschiedene Methoden aus probiert, um die Lebensdauer zu extrapolieren. 
 
\begin{figure}[H]
	\centering
	\includegraphics[width=\textwidth]{build/plot4.pdf}
	\caption{Messdaten}\label{fig:plt4}
\end{figure}

Dazu ist das Spektrum in verschiedenen Bereiche unterteilt. 

\begin{enumerate}
	\item \textbf{Der rote Bereich} sind alle Daten die nicht benutzt werden, da sie entweder nach der Suchzeit $T_{such}=\qty{10}{\us}$ sind 
	oder sie zu random sind (Die Messwerte am Anfang), das eine Betrachtung keinen Nutzen bringt. 
	\item \textbf{Der blaue Bereich} ist deutlich höher als alle anderen Messwerte und weist einen steilen Abfall auf. ($\qty{0.533}{\us}<t_B<\qty{1.075}{\us}$)
	\item \textbf{Der cyane Bereich} wird durch ein Flachen linearen Abfall gekennzeichnet, 
	welcher am ehesten von der Messrate her auf Myonenzerfall hindeuten könnte. ($\qty{1.075}{\us}<t_B<\qty{10}{\us}$)
\end{enumerate}

\subsubsection{Der Cyan Bereich}
Der Bereich mit der ungewöhnlichen Kante wird ignoriert, da dessen Zählraten deutlich über die zu erwartenden Zahlraten von Myonen sind. 
Die Zählraten haben dabei jeweils den Fehler $\sqrt(N) $, welcher in allen kommenden Fits berücksichtigt wird. 
Der zu erwartenden Verlauf von dem Zerfall von Myonen ist ein exponentieller Abfall der Zahlrate oder in einem Logarithmischen Plot einen linearen Abfall. 
Dazu kommt ein möglicher Untergrund $U$ in den Messwerten, der zu berücksichtigt werden sollte. 
Deshalb wird eine exponentielle Ausgleichsrechnung der Funktion

\begin{equation}
	f(t)=N_0 \cdot \exp(-\lambda t) + U
	\label{eqn:exp}
\end{equation}

und eine lineare Ausgleichsgerade der Form

\begin{equation}
	g(t)=m \cdot t +b
	\label{eqn:lin}
\end{equation}

mittels $curfit$ von $scipy$ ermittelt. 
Der exponentiell Fit berücksichtigt dabei einen nicht trivialen Untergrund die lineare Ausgleichsgerade geht von einem vernachlässigbaren aus. 

\begin{figure}[H]
	\centering
	\includegraphics[width=\textwidth]{build/plot5.pdf}
	\caption{Bestimmung der Lebensdauer von Myonen im cyan Bereich}\label{fig:plt5}
\end{figure}

Die Parameter des exponentiellen Fit ergeben:

\begin{align}
	\lambda&=\qty{0.602(0.018)}{\per\us} &\Rightarrow & & \tau&=\frac{1}{\lambda }=\qty{1.66(0.05)}{\us} \\
	N_0&=\num{184(8)} \notag \\
	U&=\num{1.71(0.26)} \notag
\end{align}

und die der Ausgleichsgerade: 

\begin{align}
	m&=\qty{-0.438(0.014)}{\per\us} &\Rightarrow & & \tau&=\frac{1}{|m|}=\qty{2.28(0.07)}{\us} \\
	b&=\num{4.60(0.11)} &\Rightarrow & & N_0&=\num{100(11)} \notag
\end{align}

\subsubsection{Verbindung vom Blauen und den Cyanen Bereich}
Es soll nun versucht werden die ungewöhnliche Kante zu entfernen und danach eine weitere Ausgleichsrechnung durchzuführen. 
Dazu werden alle Messwerte im blauen Bereich um die ungefähre Höhe der Kante reduziert. 
Eine Betrachtung des blauen Bereichs allein ist nicht sinnvoll, da dieser definitiv keine Myonen widerspiegelt, sondern deren Signal nur überlagert. 
Die Überlegung hinter der Reduzierung des Gebietes ist diese Überlagerung zu verkleinern, ohne den charakteristischen Verlauf des Bereichs zu verändern. 
Es wird danach wieder die zwei Ausgleichsrechnungen \eqref{eqn:exp} und \eqref{eqn:lin} durchgeführt. 

\begin{figure}[H]
	\centering
	\includegraphics[width=\textwidth]{build/plot6.pdf}
	\caption{Bestimmung der Lebensdauer von Myonen im blau und cyan Bereich}\label{fig:plt6}
\end{figure}

\newpage
Diesmal sind die Parameter des exponentiellen Fit:

\begin{align}
	\lambda&=\qty{18.31(0.015)}{\per\us} &\Rightarrow & & \tau&=\frac{1}{\lambda }=\qty{0.0546(0.0005)}{\us} \\
	N_0&=\num{1.13(0.1)e10} \notag \\
	U&=\num{5.8(1)} \notag
\end{align}

und die der Ausgleichsgerade:

\begin{align}
	m&=\qty{-0.439(0.013)}{\per\us} &\Rightarrow & & \tau&=\frac{1}{|m|}=\qty{2.28(0.07)}{\us} \\
	b&=\num{4.61(0.11)} &\Rightarrow & & N_0&=\num{101(11)} \notag
\end{align}

%---------------------------------------------------------------------------------------------------------------------------------------------------------------%
\newpage
\section{Diskussion}


%---------------------------------------------------------------------------------------------------------------------------------------------------------------%
\newpage
\printbibliography

\end{document}