\section{Zielsetzung}
In diesem Versuch wird der Effekt der Faraday Rotation verwendet um die
effektive Masse der Leitungselektronen in n-dotierten Galliumarsenid zu messen.

\section{Theorie}
In diesem Versuch scheint eine polarisierte elektromagnetische Welle durch
einen Halbleiter. Dieser Halbleiter wird einem Magnetfeld ausgesetzt was zu
einer Verdrehung der Polarisationsachse führt. Diese Begriffe werden im
folgenden definiert

\subsection{Halbleiter \cite[][Kap. 14]{book:expi3}}
Elektronen in Festkörpern haben anstelle von diskret definierten Zuständen
aufenthaltswahrscheinlichkeiten in der Form von Bändern. Leitende Festkörper
zeichnen sich durch Elektronen im ungebundenen Zustand aus. Isolatoren wiederum
haben eine große Bandlücke zwischen den Gebundenen Elektronen und dem ersten
freien Leitungsband. Diese Bandlücke ist bei Halbleitern in der Größenordnung
von etwa $qty{1}{\eV}$. Die Bandstruktur von Halbleitern ist auch eine Funktion
des Impuls der Elektronen bzw. des Wellenvektors $\vec{k}$

Vielleicht braucht man ja diese Formeln:
\begin{align}
	E                 & = E_L + (\hbar k²)/(2 m_e)                     \\
	\frac{d v_g}{d t} & = \frac{1}{\hbar^2}\frac{d^2 E}{d^2 k} \vec{F} \\
	m^*               & = \hbar^2 \cdot \frac{d²E}{d k_i d k_j^{-1}}
\end{align}
