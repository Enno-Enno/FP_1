\section{Zielsetzung}
In diesem Versuch wird der Effekt der Faraday Rotation verwendet um die
effektive Masse der Leitungselektronen in n-dotierten Galliumarsenid zu messen.

\section{Theorie}
In diesem Versuch scheint eine polarisierte elektromagnetische Welle durch
einen Halbleiter. Dieser Halbleiter wird einem Magnetfeld parallel zur
Lichtausbreitung ausgesetzt, was zu einer Drehung der Polarisationsachse führt.
Die konkreten Prozesse werden im Folgenden beschrieben.

\subsection{Halbleiter \cite[][Kap. 14]{book:expi3}}

Elektronen in Festkörpern haben anstelle von diskret definierten Zuständen
Aufenthaltswahrscheinlichkeiten in der Form von Bändern. Leitende Festkörper
zeichnen sich durch Elektronen im ungebundenen Zustand aus. Isolatoren wiederum
haben eine große Bandlücke $E_g > 3 \unit{\eV}$ \cite{web:Bandlücke} zwischen
den gebundenen Elektronen und dem ersten freien Leitungsband. Diese Bandlücke
ist bei Halbleitern in der Größenordnung von etwa
$\qty{1}{\eV}$\cite{web:Bandlücke}. Die Bandstruktur von Halbleitern ist auch
eine Funktion des Impulses der Elektronen bzw. deren Wellenvektor $\vec{k}$.
Die Energiefunktion, die das Band beschreibt, kann durch eine quadratische
Funktion angenähert werden (siehe Abbildung \ref{fig:band}).

\begin{figure}
	\centering
	\includegraphics[width=0.8\textwidth]{./Bilder/bandstrukt.png}
	\caption{Die Bandstruktur eines Halbleiters im $k$-Raum\cite{book:expi3}.}\label{fig:band}
\end{figure}

\subsection{Effektive Masse \cite[][Kap. 14]{book:expi3}}

Die effektive Masse ist relevant für die Bewegungsgleichungen der Elektronen im
Leitungsband von Halbleitern sowie der Elektronenlöcher im Valenzband. Sie
kann anstelle der Masse eines freien Elektrons verwendet werden, so können die
Bewegungsgleichungen für freie Elektronen auch in Materie angewendet werden. Sie ist
definiert als

\begin{align}
	m^* & = \hbar^2 \cdot \frac{d²E}{d k_i d k_j}
\end{align}

\subsection{Zirkulare Doppelbrechung \cite{man_a}}
\begin{figure}[H]
	\centering
	\includegraphics[width=0.8\textwidth]{./Bilder/zirpol.png}
	\caption{Rotation der Polarisationsebene bei zirkularer Doppelbrechung \cite{man_a} }\label{fig:zirpol}
\end{figure}

Es ist möglich linear polarisiertes Licht in eine
rechts und eine linkszirkular polarisierte Welle zu zerlegen.

\begin{align}
	E(z)    & = \frac{1}{2}(E_R(z) + E_L(z))                   \\
	E_R (z) & = (E_0 \vec{x}_0  - i E_0 \vec{y}_0) e^{i k_R z} \\
	E_L (z) & = (E_0 \vec{x}_0  + i E_0 \vec{y}_0) e^{i k_L z}
\end{align}


Bestimmte Kristalle können so die Polarisationsebene eines Linear polarisierten
Lichtstrahls drehen, da die Brechungsindizes für links und für rechtszirkular 
polarisiertes Licht unterschiedlich sind. 

Mit den Abkürzungen $ \psi := \frac{L}{2} (k_R + k_L)$ und $\theta :=
	\frac{L}{2} (k_R -k_L)$ ergibt sich für die Polarisationsebene am Ende des
Kristalls mit der Länge $L$
\begin{align}
	E(L) & = E_0 e^{i\psi} (\cos\theta \vec{x}_0 + \sin\theta \vec{y}_0).
\end{align}
Die Phasengeschwindigkeit der Welle kann im Allgemeinen durch die Relation $V_\t{Ph}=\omega/k$
ausgedrückt werden. Es folgt
\begin{align}
	\theta & = \frac{L \omega}{2} \left(\frac{1}{V_{Ph_R}} - \frac{1}{V_{Ph_L}}\right).
	\intertext{ Das lässt sich auch bezogen auf die Brechungsindizes mit
	der Vakuumlichtgeschwindigkeit $c$ mit $n = c/v_{Ph}$ darstellen}
	\theta & = \frac{L\omega}{2c}(n_R -n_L)
	\label{eq:kodern}
\end{align}

\subsection{Berechnung der Doppelbrechung in einem anisotropen Medium \cite{man_a}}
\label{sec:anisotrop}
Bei der Polarisation eines Kristalls $\vec{P} = \epsilon_0 \chi \vec{E}$ hat die dielektrische Suszeptibilität $\chi$
im anisotropen Fall die Form eines Tensors. Dieser ist häufig Symmetrisch und kann durch eine Hauptachsentransformation 
in eine Diagonalform gebracht werden. Im Folgenden soll Materie betrachtet werden deren $\chi$-Tensor nicht symmetrisch ist.
Dieser hat dann im einfachsten Fall die Form

\begin{align}
	\chi = %
	\begin{pmatrix}
		\chi_{xx}     & i \chi_{xy} & 0         \\
		- i \chi_{xy} & \chi_{yy}   & 0         \\
		0             & 0           & \chi_{zz}
	\end{pmatrix}.
	\label{eq:chi}
\end{align}
Verwenden der Wellengleichung $\nabla \times (\nabla \times \vec{E}) = \frac{1}{c²}(1 + \chi)\frac{d² \vec{E}}{d t²}$
mit der dielektrischen Verschiebung $\vec{D} = \epsilon_0 \vec{E}+ \vec{P}$ und einer Ebenen welle
$\vec{E}(\vec{r},t) = \vec{E}_0 e^{i(\vec{k}\vec{r} - \omega t)}$ ergibt sich eine Form
\begin{align}
	\vec{k} \times (\vec{k} \times \vec{E})= -\frac{\omega²}{c²} \vec{E}- \frac{\omega²}{c²}\chi \vec{E}.
\end{align}
Mit einem $\vec{k}$ in $z$-Richtung ergeben sich für die Wellenzahl zwei Werte
\begin{align}
	k_{\pm} = \frac{\omega}{c}\sqrt{(1+\chi_{xx})\pm \chi_{xy}}.
\end{align}
Es ergeben sich weiter zwei Phasengeschwindigkeiten mit $v = \omega/ k$
\begin{align}
	v_{\t{Ph}_\t{R}} = \frac{c}{\sqrt{1+\chi_{xx}+\chi{xy}}} \text{ und } v_{\t{Ph}_\t{L}} = \frac{c}{\sqrt{1+\chi_{xx}-\chi_{xy}}}
\end{align}
die entweder größer oder kleiner als die Phasengeschwindigkeit $v_{Ph}= \frac{c}{\sqrt{1+\chi_{xx}}}$ bei $\chi_{xy} = 0$ sind. %TODO

Der Drehwinkel lässt sich gemäß Formel \eqref{eq:kodern} berechnen
\begin{align}
	\theta = \frac{L}{2}(k_+ -k_-) = \frac{L\omega}{2c}{ \sqrt{(1+ \chi_{xx} )+ \chi_{xy}} - \sqrt{(1+ \chi_{xx} )-\chi_{xy}} }
\end{align}

Da die Werte für $\chi_{xy}$ typischerweise deutlich kleiner sind als
$\chi_{xx}$, kann dieser Ausdruck noch weiter genähert und mit der
Phasengeschwindigkeit und dem Brechungsindex $n$ vereinfacht werden.

\begin{align}
	\theta \simeq \frac{L \omega}{2c}\left\{ \sqrt{1+\chi_{xx}}\right\}^{-1} \chi_{xy} =\frac{L \omega}{2c²} v_{Ph} \chi_{xy} = \frac{L \omega}{2c n} \chi_{xy}
\end{align}

\subsection{Berechnung des Rotationswinkels der Polarisationsebene beim Faraday Effekt \cite{man_a} }
Die Bewegungsgleichung für ein gebundenes Elektron in einem Festkörper lautet

\begin{align}
	m \frac{d² \vec{r}}{d t²} + K \vec{r} = -e_0 \vec{E}(\vec{r})- e_0 \frac{d \vec{r}}{dt} \times \vec{B}.
\end{align}

$K$ ist hierbei eine Konstante, und $\vec{r}$ ist die Auslenkung des Elektrons aus seiner Ruhelage.
Aus der Beschreibung dieser Gleichung im Sinne des Verschiebungsstroms $\vec{P} = -N e_0 \vec{r}$
mit der Ladungsdichte $N$
und einem elektrischen Feld der Form $\vec{E} = e^{- i \omega t}$ ergibt sich für große $\omega$

\begin{align}
	-m\omega² \vec{P} + K \vec{P} = e_0^2 E_y -i e_0 \omega \vec{P} \times \vec{B}.
\end{align}

Der Magnetisierungstensor $\chi$ kann wie in \eqref{eq:chi} angenommen werden.
Analog zu Abschnitt \ref{sec:anisotrop} kann die Drehung des Polarisationswinkels festgestellt werden
\begin{align}
	\theta = \frac{e_0^3 \omega² NBL}{[2 \epsilon_0 c(-m\omega²+ K)² -(e_0 \omega B)²]n}.
\end{align}
Die Frequenzabhängigkeit wird noch einmal separat diskutiert.

\begin{align}
	\theta = \frac{e_0^3 \omega² NBL}{[2 \epsilon_0 c m²(-\omega²+ \frac{K}{m})² -(\frac{e_0}{m}\omega B)²]n}
\end{align}

In dieser Darstellung sind die Frequenzen $\omega_0 = \sqrt{K/m}$ und $\omega_C
	= Be_0/m$ von Bedeutung. $\omega_0$ ist eine Resonanzfrequenz. $\omega_C$ ist
die Zyklotronfrequenz, bei der ein freies Elektron im Magnetfeld einer
Kreisbahn folgen würde. Mit $B \simeq \qty{1}{\tesla}$ ergeben sich
Zyklotronfrequenzen von $\omega_C \simeq \qty{e11}{\hertz}$. Die
Resonanzfrequenzen liegen bei Halbleitern normalerweise im Infrarot ($\omega_0
	\simeq \qtyrange{e14}{e15}{\hertz}$). So ist zumeist $(\omega_0- \omega)>>
	\omega² \omega_C²$ und folgende Näherungen werden für $\omega$ mit großen (und
größerem) Abstand von $\omega_0$ möglich
\begin{align}
	\theta \simeq \frac{e_0³}{2\epsilon_0 c}\frac{1}{m²}\frac{\omega²}{(\omega_0²- \omega²)²}\frac{NBL}{n}.%
	% \simeq \frac{e_0³}{2\epsilon_0 c}\frac{1}{m²}\frac{\omega²}{\omega_0⁴}\frac{NBL}{n}.
\end{align}
Mit Wellenlänge anstelle der Kreisfrequenz wird der Fall quasifreier Ladungsträger betrachtet ($\omega \rightarrow 0$)
\begin{align}
	\theta \simeq  \frac{e_0³}{2\epsilon_0 c}\frac{1}{m²}\frac{1}{\omega²}\frac{NBL}{n}%
	= \frac{e_0³}{8\pi²\epsilon_0 c³}\frac{1}{m²}\lambda²\frac{NBL}{n}
\end{align}
Diese Gleichung bleibt auch für Kristallelektronen gültig, wenn man $m$ durch die effektive Masse $m^*$ ersetzt.
Die Endgültige Gleichung für $\theta_\t{frei} = \theta / L$ ist also
\begin{align}
	\theta_\t{frei} = \frac{e_0³}{8\pi²\epsilon_0 c³}\frac{1}{(m^*)²}\lambda²\frac{NB}{n}
	\label{eq:theta_final}
\end{align}

%% Hier kommt noch ein Teil hin um zu der verwendeten Formel zu kommen 

% Vielleicht braucht man ja diese Formeln:
% \begin{align}
% 	E                 & = E_L + \frac{\hbar k²}{2 m_e}                 \\
% 	\frac{d v_g}{d t} & = \frac{1}{\hbar^2}\frac{d^2 E}{d^2 k} \vec{F} \\
% 	E_e(\vec{k})      & = \frac{\hbar²}{3m^*_e}
% \end{align}

% \subsection{Fehlerrechnung}
% Für die Fehlerrechnung werden alle \textbf{Mittelwerte} von $N$ Messungen
% folgendermaßen berechnet:

% \begin{equation}
% 	\overline{x} = \frac{1}{N} \cdot \sum_{i=1}^N x_i
% 	\label{eqn:Mittelwert}
% \end{equation}

% und alle \textbf{Standardabweichungen zum Mittelwert} mit:

% \begin{equation}
% 	\increment\overline{x} = \sqrt{\frac{1}{N\cdot(N-1)}\cdot\sum_{i=1}^N (x_i-\overline{x})^2}
% 	\label{eqn:St_Mittelwert}
% \end{equation}

% Der Fehler für zusammenhängende Messwerte wird dann mit der \textbf{Gaußschen
% 	Fehlerfortpflanzung} berechnet:

% \begin{equation}
% 	\increment{f} = \sqrt{ \sum_{i = 1}^{N}  \biggl(\frac{\partial{f}}{\partial{x_i}}\biggr)^2\cdot(\increment{x_i})^2}
% 	\label{eqn:Gauss}
% \end{equation}

% Die Fehlerfortpflanzung wird mit Uncertainties in Python \cite{uncertainties}
% ermittelt.

%---------------------------------------------------------------------------------------------------------------------------------------------------------------%

\section{Durchführung \cite{man}} % TODO Zahlen überprüfen
Zwei Spulen werden hintereinander mit einem konstanten Strom von
$\qty{10}{\ampere}$ versorgt. Eine Hallsonde wird durch die Spule geschoben, um 
das maximale Magnetfeld zu messen. Die Proben werden im Versuch in der
Mitte der Spule eingespannt, wo das Magnetfeld am stärksten ist.

Der Versuch wird wie in Abbildung \ref{fig:aufbau} aufgebaut. Weißes Licht wird
in einem einstellbaren Polarisator polarisiert und scheint durch die Spule. Im
Zentrum der Spule wird bei dem höchsten Magnetfeld eine Probe mit einer
bekannten Elektronenlochdichte eingespannt. Das Licht fällt danach auf ein
Glen-Thomson Prisma, welches das Licht in s-polarisiertes und p-polarisiertes
Licht spaltet. Die beiden Photowiderstände messen die zwei Lichtstrahlen und
geben ihr Signal weiter an einen Differenzverstärker. Dessen Signalspannung ist
proportional zu dem Intensitätsunterschied zwischen den beiden
Polarisationsachsen. Wenn die Signalspannung des Differenzverstärkers null
(bzw. minimal) wird, ist die Polarisation genau diagonal zu dem Prisma.
Außerdem wird noch einen Wellenlängenfilter nach der Probe im Lichtstrahl
eingebaut damit die Messung des Faraday Winkels in Abhängigkeit der Wellenlänge
möglich ist.

Der Faraday Effekt rotiert die Polarisationsebene des Lichtes in der Probe um
den Winkel $\theta$. Um diesen Winkel zu messen wird, das mit einem Goniometer
einstellbare, Glen-Thomson-Prisma gedreht bis die Differenzspannung minimal
ist. Dannach wird das Magnetfeld umgepolt und das Signal erneut minimiert.
Dieses wird dann für alle $3 \, GaAS$ - Proben und für jeweils $9$
verschiedenen Wellenlängen durchgeführt.

\begin{figure}
	\centering
	\includegraphics[width=0.8\textwidth]{./Bilder/aufbau.png}
	\caption{Der Versuchsaufbau \cite{man}}\label{fig:aufbau}
\end{figure}

\newpage