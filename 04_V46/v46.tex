\input{../header.tex}

\title{V46:\\ Faraday-Effekt an Halbleitern}
\author{Benedikt Lütke Lanfer \and Enno Wellmann}
\date{10. Juni 2024}
\publishers{TU Dortmund – Fakultät Physik}

\begin{document}
\begin{titlingpage}
    \begin{center}
        \begin{Huge}
            \textbf{\thetitle\\}
        \end{Huge}
    \end{center}
    \vspace{4cm}
    \includegraphics[width=\textwidth]{Bilder/Logo_TU.png} \\
    \vspace{4cm}
    \begin{center}
        \begin{huge}
            \theauthor\\
        \end{huge}
        \vspace{0.5cm}
        \begin{Large}
            benedikt.luetkelanfer@tu-dortmund.de\\
            enno.wellmann@tu-dortmund.de\\
            \vspace{1.4cm}
            Bearbeitet: \today\\
            Abgabe: \thedate\\
            TU Dortmund – Fakultät Physik\\
        \end{Large}
    \end{center}
\end{titlingpage}
\tableofcontents
\newpage

\section{Zielsetzung}

%---------------------------------------------------------------------------------------------------------------------------------------------------------------%

\section{Theorie}

\subsection{Fehlerrechnung}
Für die Fehlerrechnung werden alle \textbf{Mittelwerte} von $N$ Messungen folgendermaßen berechnet:

\begin{equation}
    \overline{x} = \frac{1}{N} \cdot \sum_{i=1}^N x_i
    \label{eqn:Mittelwert}
\end{equation}

und alle \textbf{Standardabweichungen zum Mittelwert} mit:

\begin{equation}
    \increment\overline{x} = \sqrt{\frac{1}{N\cdot(N-1)}\cdot\sum_{i=1}^N (x_i-\overline{x})^2}
    \label{eqn:St_Mittelwert}
\end{equation}

Der Fehler für zusammenhängende Messwerte wird dann mit der \textbf{Gaußschen Fehlerfortpflanzung} berechnet:

\begin{equation}
    \increment{f} = \sqrt{ \sum_{i = 1}^{N}  \biggl(\frac{\partial{f}}{\partial{x_i}}\biggr)^2\cdot(\increment{x_i})^2}
    \label{eqn:Gauss}
\end{equation}

Die Fehlerfortpflanzung wird mit Uncertainties in Python \cite{uncertainties} ermittelt.

%---------------------------------------------------------------------------------------------------------------------------------------------------------------%

\section{Durchführung}


%---------------------------------------------------------------------------------------------------------------------------------------------------------------%
\newpage
\section{Auswertung}
\subsection{Bestimmung des B-Feldes}
Die Messung des B-Feldes mittels einer Hall-Sonde ergibt die in Tabelle \eqref{tab:B_Feld} folgende Werte:

\begin{table}[H]
	\centering
	\begin{tabular}{c c}
		\toprule
		$d \, [\unit{\milli\meter}]$ & $B \, [\unit{\milli\tesla}] $  \\
		\midrule
        170 & 1 \\
        160 & 0 \\
        150 & 0 \\
        140 & 0 \\
        130 & 1 \\
        120 & 6 \\
        112 & 92 \\
        110 & 176 \\
        108 & 280 \\
        105 & 382 \\
        102 & 418 \\
        100 & 426 \\
        98  & 428 \\
        95  & 420 \\
        90  & 340 \\
        87  & 236 \\
        85  & 97 \\
        80  & 17 \\
        70  & 0  \\
		\bottomrule
	\end{tabular}
    \label{tab:B_Feld}
\end{table}

Diese Werte werden einmal geplottet \eqref{fig:B_Feld}, welches eine gut sichtbare Parabel förmige Struktur ergibt. 
Jedoch wurden die Werte von $\qty{140}{\milli\meter}$ bis $\qty{170}{\milli\meter}$ ausgelassen das diese Vernachlässigt werden können. 
Das maximale B-Feld ergibt sich dabei mit $\qty{428}{\milli\tesla}$ bei $\qty{98}{\milli\meter}$. 

\begin{figure}[H]
	\centering
	\includegraphics[width=\textwidth]{build/B_Feld.pdf}
	\caption{Messdaten B-Feld}\label{fig:B_Feld}
\end{figure}

\subsection{Bestimmung der effektiven Masse}
Die Differenz der einzelnen aufgenommen Messwerte der Winkel je nach B-Feld Ausrichtung werden zuerst mittels 

\begin{equation}
    \theta=\frac{\theta_2 - \theta_1}{2L} 
\end{equation}

auf die Länge $L$ normiert. 
Der Faktor $2$ kommt daher, dass die Differenz des B-Feldes doppelt so stark ist wie das vorher bestimmte maximale B-Feld ist. 

\begin{table}[H]
	\centering
	\caption{Ergebnisse Vollenergienachweiswahrscheinlichkeit}
	\begin{tabular}{|c c | c c | c c|}
		\toprule
		\multicolumn{2}{|c|}{$GaAs_n$}  & \multicolumn{2}{|c|}{$GaAs_r \, $}  & \multicolumn{2}{|c|}{$GaAs_n \, $}   \\
        \multicolumn{2}{|c|}{$N=\qty[per-mode=fraction]{2.8e18}{\per\cubic\centi\meter} $}  & \multicolumn{2}{|c|}{$ $}  & \multicolumn{2}{|c|}{$N=\qty[per-mode=fraction]{1.28e18}{\per\cubic\centi\meter} $}   \\
        \midrule
        $\theta_1$ & $\theta_2 $ & $\theta_1 $ & $ \theta_2 $ & $\theta_1 $ & $ \theta_2 $ \\
		\midrule
            \num{77.50} & \num{88.60} & \num{70.50} & \num{93.50} & \num{78.42} & \num{87.05} \\
            \num{78.00} & \num{86.12} & \num{75.08} & \num{90.42} & \num{80.17} & \num{86.93} \\
            \num{74.17} & \num{87.63} & \num{78.50} & \num{90.50} & \num{80.25} & \num{88.08} \\
            \num{77.08} & \num{86.48} & \num{78.53} & \num{87.00} & \num{80.00} & \num{86.75} \\
            \num{71.33} & \num{87.27} & \num{74.83} & \num{82.50} & \num{75.33} & \num{81.08} \\
            \num{69.50} & \num{82.32} & \num{74.67} & \num{79.17} & \num{72.78} & \num{78.83} \\
            \num{44.08} & \num{51.50} & \num{48.50} & \num{52.50} & \num{45.25} & \num{50.25} \\
            \num{27.50} & \num{40.41} & \num{29.17} & \num{32.33} & \num{28.25} & \num{36.00} \\
            \num{63.75} & \num{78.22} & \num{67.83} & \num{73.17} & \num{66.60} & \num{73.08} \\
		\bottomrule
	\end{tabular}
	\label{tab:data2}
\end{table}

\begin{figure}[H]
	\centering
	\includegraphics[width=\textwidth]{build/Messdaten.pdf}
	\caption{Messdaten Proben}\label{fig:Aufbau}
\end{figure}

\begin{figure}[H]
	\centering
	\includegraphics[width=\textwidth]{build/plot1.pdf}
	\caption{Ausgleichsgerade}\label{fig:Aufbau}
\end{figure}

\begin{figure}[H]
	\centering
	\includegraphics[width=\textwidth]{build/plot2.pdf}
	\caption{Ausgleichsgerade}\label{fig:Aufbau}
\end{figure}

%---------------------------------------------------------------------------------------------------------------------------------------------------------------%

\section{Diskussion}


%---------------------------------------------------------------------------------------------------------------------------------------------------------------%
\newpage
\printbibliography

\end{document}