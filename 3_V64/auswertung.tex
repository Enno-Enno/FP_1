\section{Evaluation}

%Alignment of the Laser paths

\subsection{Optimal polarization adjustment}
In tabular ... the minimal and maximal Energies recorded by the photodiode are
noted. The original measured values can be found in the appendix in tabular ...
the mean values and their standard deviations are calculated. For Intensity
standard deviations less than the digital reading error of \qty{0.005}{\volt}
the error of \qty{0.005}{\volt} is assumed.


\begin{table}[H]
	\centering
	\sisetup{table-align-uncertainty}
	\begin{tabular}{S S S S S}
		{Polarization/\unit{\degree}} & {$I_\text{max}/\unit{\volt}$} & {$I_\text{min}/\unit{\volt}$} & {Contrast}                       \\
		0                             & 1.600 (0.022)                 & 1.357 (0.005)                 & 0.082 (0.007 ) & 0.082 (0.007 )  \\
		10                            & 1.327 (0.005)                 & 0.950 (0.008)                 & 0.165 (0.005 ) & 0.165 (0.005 )  \\
		20                            & 1.440 (0.016)                 & 0.453 (0.005)                 & 0.521 (0.006 ) & 0.521 (0.006 )  \\
		25                            & 1.333 (0.025)                 & 0.300 (0.008)                 & 0.633 (0.010 ) & 0.633 (0.010 )  \\
		30                            & 1.203 (0.033)                 & 0.217 (0.017)                 & 0.695 (0.021 ) & 0.695 (0.021 )  \\
		35                            & 1.160 (0.005)                 & 0.177 (0.005)                 & 0.736 (0.007 ) & 0.736 (0.007 )  \\
		40                            & 1.150 (0.022)                 & 0.120 (0.005)                 & 0.811 (0.008 ) & 0.811 (0.008 )  \\
		45                            & 1.290 (0.022)                 & 0.100 (0.005)                 & 0.856 (0.007 ) & 0.856 (0.007 )  \\
		50                            & 1.363 (0.005)                 & 0.103 (0.005)                 & 0.859 (0.006 ) & 0.859 (0.006 )  \\
		55                            & 1.500 (0.008)                 & 0.120 (0.005)                 & 0.852 (0.006 ) & 0.852 (0.006 )  \\
		60                            & 1.617 (0.009)                 & 0.180 (0.005)                 & 0.800 (0.005 ) & 0.800 (0.005 )  \\
		70                            & 2.010 (0.022)                 & 0.417 (0.005)                 & 0.657 (0.005 ) & 0.657 (0.005 )  \\
		80                            & 2.153 (0.025)                 & 0.920 (0.005)                 & 0.401 (0.005 ) & 0.401 (0.005 )  \\
		90                            & 2.200 (0.008)                 & 1.613 (0.009)                 & 0.1538(0.0034) & 0.1538 (0.0034) \\
		100                           & 2.673 (0.024)                 & 1.683 (0.012)                 & 0.227 (0.005 ) & 0.227 (0.005 )  \\
		110                           & 3.94  (0.05 )                 & 1.077 (0.017)                 & 0.570 (0.007 ) & 0.570 (0.007 )  \\
		120                           & 4.68  (0.10 )                 & 0.657 (0.025)                 & 0.754 (0.009 ) & 0.754 (0.009 )  \\
		130                           & 4.94  (0.09 )                 & 0.383 (0.017)                 & 0.856 (0.006 ) & 0.856 (0.006 )  \\
		140                           & 5.11  (0.05 )                 & 0.430 (0.008)                 & 0.8449(0.0031) & 0.8449 (0.0031) \\
		150                           & 4.28  (0.10 )                 & 0.64  (0.04 )                 & 0.740 (0.014 ) & 0.740 (0.014 )  \\
		160                           & 3.39  (0.15 )                 & 0.867 (0.017)                 & 0.593 (0.016 ) & 0.593 (0.016 )  \\
		170                           & 2.35  (0.09 )                 & 1.163 (0.024)                 & 0.337 (0.019 ) & 0.337 (0.019 )  \\
		180                           & 1.663 (0.005)                 & 1.387 (0.005)                 & 0.0907(0.0023) & 0.0907 (0.0023) \\
        
	\end{tabular}

\end{table}

The contrast ist calculated with
\begin{align}
    \text{contrast} = \frac{I_\text{max} -I_\text{min}}{I_\text{max}-I_\text{min}}
\end{align}
Based on formula ... for the maximal and minimal intensities one expects the contrast to have the form
\begin{align}
    \text{contrast}(\theta) = 2\cos(theta)\sin(theta) 
\end{align}



