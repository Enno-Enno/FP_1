\input{../header.tex}

\title{V64: Modern Interferometry}
\author{Benedikt Lütke Lanfer \and Enno Wellmann}
\date{27. Mai 2024}
\publishers{TU Dortmund – Fakultät Physik}

\begin{document}
\begin{titlingpage}
    \begin{center}
        \begin{Huge}
            \textbf{\thetitle\\}
        \end{Huge}
    \end{center}
    \vspace{4cm}
    \includegraphics[width=\textwidth]{Bilder/Logo_TU.png} \\
    \vspace{4cm}
    \begin{center}
        \begin{huge}
            \theauthor\\
        \end{huge}
        \vspace{0.5cm}
        \begin{Large}
            benedikt.luetkelanfer@tu-dortmund.de\\
            enno.wellmann@tu-dortmund.de\\
            \vspace{1.4cm}
            Bearbeitet: \today\\
            Abgabe: \thedate\\
            TU Dortmund – Fakultät Physik\\
        \end{Large}
    \end{center}
\end{titlingpage}
\tableofcontents
\newpage
%Shortcuts
\let\t\text\


% \section{Vorbereitung}
$\Delta t_c  = 1/ \Delta \nu$
Kohärenzzeit ist die Zeitspanne in der man die Phase der Lichtwelle noch vorhersagen kann
Erzeigung von Interferenzmustern ist ein gutes Maß für Kohärenz.
Interferenztherm
$E^2 = (E_1 +E_2)^2$

$$I_{12} = E_{01}E_{02}cos{\delta} = 2 \sqrt{I_1 I_2} \cos{\delta}$$
$\delta = (k_1 r- k_2 r + \epsilon_1-\epsilon_2)$
Auslöschung bei $\delta= 0, \pm 2\pi,\pm4\pi,\dots$

Lorenz-Lorentz-Gesetz
Annäherung für Gase:
$$n \simeq \sqrt{1 + \frac{3A_p}{R T}} $$


\section{Objective}
Goal of this Experiment is to learn the function of a Sagnac Interferometer and to measure its contrast as well as the 
refractive index of glass and air, dependent of the pressure. 

%---------------------------------------------------------------------------------------------------------------------------------------------------------------%

\section{Theory}
In the experiment a Helium-Neon-Laser, which emits linear polarized coherent light, is used in a Sagnac Interferometer.
To understand the experiment properly the following theory will explain the necessary technical terms as well as the used setup. 

\subsection{Coherence}
Coherence describes weather light is temporally or spatial in Phase. 
If light is temporally coherent than its Phase difference is constant for a time period $\Delta t_c$, 
equally if light is spatial coherent than its Phase difference is constant in a limited space. 
As a result of superimposing two coherent waves, spatial and temporally Interference can be observed.

\subsection{Polarization}
Another characteristic of light is its polarization, which describes the geometrical orientation of the oscillations of the light. 
Light consists of perpendicular oscillating electric and magnetic fields.
Normally for polarization only one field oscillations is depicted for simplicity, because the fields always stands perpendicular to each other and
therefore have the same polarization.
The three common polarization types are

\begin{enumerate}
    \item linear polarization: oscillations on a line 
    \item circular polarization: oscillations on a circle
    \item elliptic polarization: oscillations on an ellipse
\end{enumerate}

which are better shown in the following graphic

\begin{figure}[H]
	\centering
	\includegraphics[width=\textwidth]{Bilder/Polarization.png}
	\caption{Polarization types \cite{...}}\label{fig:po}
\end{figure}

The polarization influences weather interference pattern occur or not.
To be able to observe interference the waves have to be coherent and have the same polarization. 
In the following experiment only linear polarized light is used.
To change the direction of oscillation of the linear light a polarization filter, which only lets light oscillating in one direction trough, can be used.  
If linear light oscillating in an angle $\alpha $ to the polarization filter, the intensity of the light decreases by

\begin{equation}
    I=I_0 \cdot \cos^2(\alpha )
\end{equation}

after passing the filter. 

\subsection{Refractive Index}
The refractive index $n$ describes the velocity of the light in matter

\begin{equation}
    c_m=\frac{c}{n}
\end{equation}

compared to constant vacuum velocity $c$. 
Also,, the phase of the light change in matter by constant phase $\varphi $. 

\subsubsection{Refractive index of glass}
To estimate the refractive index of glass in the experiment two rotatable glass planes get inserted in the light beam. 
Each of the plane changes the phase of the light dependent of the rotation angel $\theta $ by

\begin{equation}
    \Delta \varphi(\theta ) =\frac{2\pi }{\lambda_{vac}}T \cdot \left(\frac{n-1}{2n}\theta ^2 +O(\theta ^4)  \right)
\end{equation}

With the wavelength of the laser $\lambda_{vac}$ and the thickness $T$.
Together the two glass planes, which are each tilted by $\varphi_0=\pm 10°$, change the phase by 

\begin{align}
    \Delta \varphi(\theta ) &=\frac{2\pi T}{\lambda_{vac}} \frac{n-1}{2n} \cdot \left[(\theta+\varphi_0)^2 - (\theta-\varphi_0)^2 +O(\theta ^4)  \right] \notag\\
                            &=\frac{4\pi T}{\lambda_{vac}} \frac{n-1}{n} \cdot \theta\varphi_0 +O(\theta ^4)
\end{align}

The phase shift is connected to the number of interference maxima and minima by 

\begin{equation}
    M=\frac{\Delta \varphi(\theta)}{2\pi}
\end{equation}

\subsubsection{Refractive index of gas}
To determine the refractive index of gas a pressure control tube of the gas is inserted in the beam.
The refractive index of gas is dependent of the temperature $T$ and the pressure $p$ of the gas by the Lorentz-Lorenz-Law

\begin{equation}
    \frac{n^2-1}{n^2+1}=\frac{Ap}{RT}
\end{equation}

were $R$ is the gas constant and $A$ is the molar refractive.
Furthermore, the light phase is shifted in the tube of length $L$ by

\begin{equation}
    \Delta \varphi(\theta ) =\frac{2\pi }{\lambda_{vac}}\Delta n L
\end{equation}

\subsection{Contrast}
The contrast of an interferometer 

\begin{equation}
    C=\frac{I_{max}-I_{min}}{I_{max}+I_{min}}
\end{equation}

describes the normalized difference in the maximal intensity $I_{max}$ and the minimal intensity $I_{min}$ and therefore $C \in [0,1] $.
The superposition of two waves with electric field components $E_1$, $E_2$ and a phase shift $\varphi $ result in an intensity of

\begin{equation}
    I\varpropto \langle |E_1 \cos(\omega t) + E_2 \cos(\omega t+\varphi ) | \rangle 
\end{equation}.

In the experiment the laser beam is split in two, therefore the absolute value of the electric field components of the two waves are the same $E_1^2=E_2^2$.
Only different in there polarization, so you can write $E_1=\frac{1}{2} \sqrt{I_{laser}} \cdot \cos(\phi )$ and $E_2=\frac{1}{2} \sqrt{I_{laser}} \cdot \sin(\phi )$, 
which result in 

\begin{equation}
    I_{\frac{max}{min}}\varpropto I_{laser}\cdot [1\pm 2\cos(\phi )\sin(\phi )]
\end{equation}

Finally, inserting this in the formula for the contrast yield

\begin{equation}
    C=\sin(2\phi )
\end{equation}


\subsection{Error calculation}
For error calculation, all \textbf{mean values} of N measurements are calculated as follows:

\begin{equation}
    \overline{x} = \frac{1}{N} \cdot \sum_{i=1}^N x_i
    \label{eqn:Mittelwert}
\end{equation}

and all \textbf{standard deviations of the mean} with:

\begin{equation}
    \increment\overline{x} = \sqrt{\frac{1}{N\cdot(N-1)}\cdot\sum_{i=1}^N (x_i-\overline{x})^2}
    \label{eqn:St_Mittelwert}
\end{equation}

The error for correlated measurements is then calculated using the \textbf{Gaussian error propagation}:

\begin{equation}
    \increment{f} = \sqrt{ \sum_{i = 1}^{N}  \biggl(\frac{\partial{f}}{\partial{x_i}}\biggr)^2\cdot(\increment{x_i})^2}
    \label{eqn:Gauss}
\end{equation}

Error propagation is determined using the Uncertainties \cite{uncertainties} package in Python.

%---------------------------------------------------------------------------------------------------------------------------------------------------------------%

\section{Experimental Procedure}

\subsection{Sagnac-Interferometer}


%---------------------------------------------------------------------------------------------------------------------------------------------------------------%

\section{Evaluation}

%Alignment of the Laser paths

\subsection{Optimal polarization adjustment}
In tabular ... the minimal and maximal Energies recorded by the photodiode are
noted. The original measured values can be found in the appendix in tabular ...
the mean values and their standard deviations are calculated. For Intensity
standard deviations less than the digital reading error of \qty{0.005}{\volt}
the error of \qty{0.005}{\volt} is assumed.


\begin{table}[H]
	\centering
	\sisetup{table-align-uncertainty}
	\begin{tabular}{S S S S S}
		\toprule
		{Polarization/\unit{\degree}} & {$I_\text{max}/\unit{\volt}$} & {$I_\text{min}/\unit{\volt}$} & {Contrast}                       \\
		\midrule
		0                             & 1.600 (0.022)                 & 1.357 (0.005)                 & 0.082 (0.007 ) & 0.082 (0.007 )  \\
		10                            & 1.327 (0.005)                 & 0.950 (0.008)                 & 0.165 (0.005 ) & 0.165 (0.005 )  \\
		20                            & 1.440 (0.016)                 & 0.453 (0.005)                 & 0.521 (0.006 ) & 0.521 (0.006 )  \\
		25                            & 1.333 (0.025)                 & 0.300 (0.008)                 & 0.633 (0.010 ) & 0.633 (0.010 )  \\
		30                            & 1.203 (0.033)                 & 0.217 (0.017)                 & 0.695 (0.021 ) & 0.695 (0.021 )  \\
		35                            & 1.160 (0.005)                 & 0.177 (0.005)                 & 0.736 (0.007 ) & 0.736 (0.007 )  \\
		40                            & 1.150 (0.022)                 & 0.120 (0.005)                 & 0.811 (0.008 ) & 0.811 (0.008 )  \\
		45                            & 1.290 (0.022)                 & 0.100 (0.005)                 & 0.856 (0.007 ) & 0.856 (0.007 )  \\
		50                            & 1.363 (0.005)                 & 0.103 (0.005)                 & 0.859 (0.006 ) & 0.859 (0.006 )  \\
		55                            & 1.500 (0.008)                 & 0.120 (0.005)                 & 0.852 (0.006 ) & 0.852 (0.006 )  \\
		60                            & 1.617 (0.009)                 & 0.180 (0.005)                 & 0.800 (0.005 ) & 0.800 (0.005 )  \\
		70                            & 2.010 (0.022)                 & 0.417 (0.005)                 & 0.657 (0.005 ) & 0.657 (0.005 )  \\
		80                            & 2.153 (0.025)                 & 0.920 (0.005)                 & 0.401 (0.005 ) & 0.401 (0.005 )  \\
		90                            & 2.200 (0.008)                 & 1.613 (0.009)                 & 0.1538(0.0034) & 0.1538 (0.0034) \\
		100                           & 2.673 (0.024)                 & 1.683 (0.012)                 & 0.227 (0.005 ) & 0.227 (0.005 )  \\
		110                           & 3.94  (0.05 )                 & 1.077 (0.017)                 & 0.570 (0.007 ) & 0.570 (0.007 )  \\
		120                           & 4.68  (0.10 )                 & 0.657 (0.025)                 & 0.754 (0.009 ) & 0.754 (0.009 )  \\
		130                           & 4.94  (0.09 )                 & 0.383 (0.017)                 & 0.856 (0.006 ) & 0.856 (0.006 )  \\
		140                           & 5.11  (0.05 )                 & 0.430 (0.008)                 & 0.8449(0.0031) & 0.8449 (0.0031) \\
		150                           & 4.28  (0.10 )                 & 0.64  (0.04 )                 & 0.740 (0.014 ) & 0.740 (0.014 )  \\
		160                           & 3.39  (0.15 )                 & 0.867 (0.017)                 & 0.593 (0.016 ) & 0.593 (0.016 )  \\
		170                           & 2.35  (0.09 )                 & 1.163 (0.024)                 & 0.337 (0.019 ) & 0.337 (0.019 )  \\
		180                           & 1.663 (0.005)                 & 1.387 (0.005)                 & 0.0907(0.0023) & 0.0907 (0.0023) \\
        \bottomrule
	\end{tabular}
	\caption{Intensities and contrast for different polarization angles}
\end{table}

The contrast ist calculated with
\begin{align}
    \text{contrast} = \frac{I_\text{max} -I_\text{min}}{I_\text{max}-I_\text{min}}
\end{align}
Based on formula ... for the maximal and minimal intensities one expects the contrast to have the form
\begin{align}
    \text{contrast}(\theta) = 2\cos(\theta)\sin(\theta) 
\end{align}

\subsection{Refractive index of glass}
\textbf{The following part could go into theory}
When changing the angle of attack $\theta$ for the glass panels the phase shift $\Delta\Phi$ can be approximated for small angles
as 
$$\Delta\Phi = 2* \frac{2\pi}{\lambda_\text{vac}} T(\theta)\left(\frac{n-1}{2n} \theta^2 \right)$$.
The length of the way of travel for the glass $T$ can be calculated with
$$T(\theta) = \frac{T_0}{\cos\theta}$$

Rearranging this formula makes it possible to calculate $n$ from the accumulated phase shift $\Delta \Phi$
\begin{align}
	\Delta \Phi (\qty{10}{\deg})- \qty{0}{\deg} =\Delta \Phi (\qty{10}{\deg}) \\
	n (\Delta \Phi) = \frac{1}{1-\frac{\Delta\Phi}{2\pi} \frac{\lambda_\text{vac}\cos{\theta}}{T_0 \theta^2}}
\end{align}
In the calculation all angle vales are converted into RAD Format.
In the measurement the number of full relative phase rotations 
$k = \frac{\Delta Phi}{2 \pi}$ was counted.






%---------------------------------------------------------------------------------------------------------------------------------------------------------------%

\section{Discussion}


%---------------------------------------------------------------------------------------------------------------------------------------------------------------%
\newpage
\printbibliography

\end{document}