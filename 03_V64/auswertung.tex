\section{Evaluation}

%Alignment of the Laser paths

\subsection{Optimal polarization adjustment}
In tabular ref{tab:contrast} the minimal and maximal Energies recorded by the
photodiode are noted and the mean values and their standard deviations calculated. 
The original measured values can be found in the appendix.
For Intensity standard deviations less than the digital reading error of
$\qty{0.005}{\volt}$ the error of $ \qty{0.005}{\volt}$ is assumed.

\begin{table}[H]
	\centering
	\sisetup{table-align-uncertainty,table-format=1.3(3)}
	\begin{tabular}{S[table-format=3.0] S S S[table-format=1.4(2)]}
		\toprule
		{Polarization$/\unit{\degree}$} & {$I_\t{max}/\unit{\volt}$} & {$I_\t{min}/\unit{\volt}$} & {Contrast}     \\
		\midrule
		0                               & 1.600 (0.022)              & 1.357 (0.005)              & 0.082 (0.007 ) \\
		10                              & 1.327 (0.005)              & 0.950 (0.008)              & 0.165 (0.005 ) \\
		20                              & 1.440 (0.016)              & 0.453 (0.005)              & 0.521 (0.006 ) \\
		25                              & 1.333 (0.025)              & 0.300 (0.008)              & 0.633 (0.010 ) \\
		30                              & 1.203 (0.033)              & 0.217 (0.017)              & 0.695 (0.021 ) \\
		35                              & 1.160 (0.005)              & 0.177 (0.005)              & 0.736 (0.007 ) \\
		40                              & 1.150 (0.022)              & 0.120 (0.005)              & 0.811 (0.008 ) \\
		45                              & 1.290 (0.022)              & 0.100 (0.005)              & 0.856 (0.007 ) \\
		50                              & 1.363 (0.005)              & 0.103 (0.005)              & 0.859 (0.006 ) \\
		55                              & 1.500 (0.008)              & 0.120 (0.005)              & 0.852 (0.006 ) \\
		60                              & 1.617 (0.009)              & 0.180 (0.005)              & 0.800 (0.005 ) \\
		70                              & 2.010 (0.022)              & 0.417 (0.005)              & 0.657 (0.005 ) \\
		80                              & 2.153 (0.025)              & 0.920 (0.005)              & 0.401 (0.005 ) \\
		90                              & 2.200 (0.008)              & 1.613 (0.009)              & 0.1538(0.0034) \\
		100                             & 2.673 (0.024)              & 1.683 (0.012)              & 0.227 (0.005 ) \\
		110                             & 3.94  (0.05 )              & 1.077 (0.017)              & 0.570 (0.007 ) \\
		120                             & 4.68  (0.10 )              & 0.657 (0.025)              & 0.754 (0.009 ) \\
		130                             & 4.94  (0.09 )              & 0.383 (0.017)              & 0.856 (0.006 ) \\
		140                             & 5.11  (0.05 )              & 0.430 (0.008)              & 0.8449(0.0031) \\
		150                             & 4.28  (0.10 )              & 0.64  (0.04 )              & 0.740 (0.014 ) \\
		160                             & 3.39  (0.15 )              & 0.867 (0.017)              & 0.593 (0.016 ) \\
		170                             & 2.35  (0.09 )              & 1.163 (0.024)              & 0.337 (0.019 ) \\
		180                             & 1.663 (0.005)              & 1.387 (0.005)              & 0.0907(0.0023) \\
		\bottomrule
	\end{tabular}
	\caption{Intensities and contrast for different polarization angles}\label{tab:contrast}
\end{table}

The contrast 
\begin{align}
	C = \frac{I_\t{max} -I_\t{min}}{I_\t{max}-I_\t{min}}
\end{align}
is calculated dependent on the angel of polarization with the formula \eqref{eq:contrast_phi}.
%Based on formula \eqref{eq:contrast} for the maximal and minimal intensities one expects the
%contrast to have the form
%\begin{align}
%	\t{contrast}(\theta) = 2\cos(\theta)\sin(\theta)
%\end{align}

\subsection{Refractive index of glass}
% \textbf{The following part could go into theory}
% When changing the angle of attack $\theta$ for the glass panels the phase shift $\Phi$ can be approximated for small angles
% as
% \begin{align}
% \Phi      & = 2 \cdot \frac{2\pi}{\lambda_\t{vac}} T(\theta)\left(\frac{n-1}{2n} \theta^2 \right) \\
% T(\theta) & = \frac{T_0}{\cos\theta}
% \end{align}
% The length of the way of light in the Glass
% It is useful to separate the equation dependent on $\theta$ into its own function
% \begin{align}
% R(\theta)= 2\cdot \frac{2\pi}{\lambda_\t{vac}} T(\theta)\cdot \theta^2
% \end{align}
% so that we can deal with the differently angled Glass panels.
% They have a starting angle of $\theta_{01} = 10°$ and $\theta_{02}= -10°$
% which gets added to the adjusted angle $\theta$.
% This way we get a combined function
% \begin{align}
% R_\t{comb} (\theta) = R(\theta_{01}+\theta) + R(\theta_{02}+\theta) \\
% \Phi = R_\t{comb}(\theta) \frac{n-1}{2n}
% \end{align}
We measure a change in phase shift $\Delta \Phi$ by counting the number of full
relative phase rotations $2 \pi M = \Delta \Phi$. From there we obtain $n$ via
equation \eqref{eq:glass}
\begin{align}
	R_\t{comb}(\theta) &= \frac{4\pi T}{\lambda_\t{vac}}\theta\phi_0\\
	n & = \frac{1}{1-\frac{\Delta \Phi}{\left|R_\t{comb}(\theta_{2})-R_\t{comb}(\theta_{1})\right|} } \label{eq:nAtmos}
\end{align}

We measured the following Values for $\theta_1 = 0°$ and $\theta_2 = 10°$
\begin{table}[H]
	\centering
	\begin{tabular}{c}
		\toprule
		Number of phase rotations \\
		\midrule
		35                        \\
		31                        \\
		30                        \\
		37                        \\
		38                        \\
		37                        \\
		37                        \\
		34                        \\
		33                        \\
		37                        \\
		34                        \\
		\bottomrule
	\end{tabular}
\end{table}

This results in a count of $\num{34.8(26)}$ and a measured diffraction index of
$n = \num{1.57(7)}$

\subsection{Refractive index of Air}
The light beam undergoes a phase shift of
\begin{align}
	\Delta\Phi = \frac{2 \pi}{\lambda_{vac} \Delta n L }
\end{align}
Where $\Delta n = n_\t{air}- n_\t{vac} = n_\t{air}-1 $.

In table \ref{tab:air} the refractive indices are calculated with the vacuum
chamber length of $\qty{100.0(1)}{\mm}$. The calculated uncertainties
underestimate the actual uncertainties because the phase shift was included
without the reading error of $\pm \pi$. This is more relevant for the low
pressures where the count is low and averaging effects of the measurement are
not yet that relevant.

\begin{table}[H]
	\centering
	\sisetup{table-format=2.0}
	\begin{tabular}{S[table-format=3.0] S S S S S[table-format = 1.9(2)] S[table-format = 1.9(2)]}
		\toprule
		                           & \multicolumn{4}{c}{measurement number}                                                             \\
		{$ p / \unit{\milli\bar}$} & {\#1}                                  & {\#2} & {\#3} & {\#4} & {$\Delta n$}    & {$n$}           \\
		\midrule
		{$\simeq 0.0$}             & 0                                      & 0     & 0     & 0     & 0.0             & 1.0             \\
		50                         & 2                                      & 2     & 2     & 2     & 0.000012660(13) & 1.000012660(13) \\
		100                        & 4                                      & 4     & 4     & 4     & 0.000025320(25) & 1.000025320(25) \\
		150                        & 6                                      & 6     & 6     & 6     & 0.00003798(4)   & 1.00003798(4)   \\
		200                        & 8                                      & 8     & 9     & 8     & 0.0000522(27)   & 1.0000522(27)   \\
		250                        & 10                                     & 11    & 10    & 11    & 0.0000665(32)   & 1.0000665(32)   \\
		300                        & 15                                     & 13    & 13    & 13    & 0.000085(5)     & 1.000085(5)     \\
		350                        & 17                                     & 15    & 15    & 15    & 0.000098(5)     & 1.000098(5)     \\
		400                        & 19                                     & 17    & 17    & 17    & 0.000111(5)     & 1.000111(5)     \\
		450                        & 21                                     & 19    & 19    & 19    & 0.000123(5)     & 1.000123(5)     \\
		500                        & 22                                     & 21    & 21    & 21    & 0.0001345(27)   & 1.0001345(27)   \\
		550                        & 25                                     & 23    & 23    & 23    & 0.000149(5)     & 1.000149(5)     \\
		600                        & 27                                     & 25    & 25    & 25    & 0.000161(5)     & 1.000161(5)     \\
		650                        & 29                                     & 28    & 27    & 27    & 0.000176(5)     & 1.000176(5)     \\
		700                        & 31                                     & 30    & 30    & 30    & 0.0001915(27)   & 1.0001915(27)   \\
		750                        & 34                                     & 33    & 32    & 32    & 0.000207(5)     & 1.000207(5)     \\
		800                        & 36                                     & 35    & 34    & 34    & 0.000220(5)     & 1.000220(5)     \\
		850                        & 38                                     & 37    & 36    & 36    & 0.000233(5)     & 1.000233(5)     \\
		900                        & 40                                     & 39    & 38    & 38    & 0.000245(5)     & 1.000245(5)     \\
		950                        & 42                                     & 41    & 40    & 40    & 0.000258(5)     & 1.000258(5)     \\
		{$\simeq $1000.0}          & 44                                     & 43    & 42    & 42    & 0.000271(5)     & 1.000271(5)     \\
		\bottomrule
	\end{tabular}
	\caption{Measured counts and values for the refractive index of air.}\label{tab:air}
\end{table}

The central data point from table \ref{tab:air} is the $\Delta n$ at
atmospheric pressure $\Delta n = 0.000271\pm0.000005$. In figure \ref{fig:air}
the measured values are compared to a linear fit with the parameters
\begin{align}
	\Delta n &= mp +b \\
	m &= \qty{2.735(17)d-07}{\per\milli\bar}\\
	b &= \num{-0.7(1.0)d-06}
\end{align}
The resulting refractive index at a standard atmospheric pressure of
$p_\t{atm}=\qty{1013.25}{\milli\bar}$ turns out to be
\begin{align}
	n_\t{fit} = \num{1.0002765 (10)}
\end{align}

\begin{figure}
	\centering
	\includegraphics[width=0.8\textwidth]{./build/plot2.pdf}
	\caption{Measured $\Delta n$ with linear fit}\label{fig:air}
\end{figure}

% According to \cite{atc:2011AmJP} the mean polarizability of air 